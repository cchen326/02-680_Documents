%%%%%%%%%%%%%%%%%%%%%%%%%%%%%%%%%%%%
%%%%%%%%%%%%%%%%%%%%%%%%%%%%%%%%%%%%
\section{Academic Integrity}
%%%%%%%%%%%%%%%%%%%%%%%%%%%%%%%%%%%%
%%%%%%%%%%%%%%%%%%%%%%%%%%%%%%%%%%%%
\label{sec:cheating}

(See: {\footnotesize\url{https://www.cmu.edu/policies/student-and-student-life/academic-integrity.html}})

Students in all CMU programs, 
because they are members of an academic community dedicated to the achievement of excellence, 
are expected to meet the highest standards of personal, ethical, and moral conduct possible.
These standards require personal integrity, 
a commitment to honesty without compromise, 
as well as truth without equivocation, 
and a willingness to place the good of the community above the good of the self. 
Obligations once undertaken must be met, commitments kept.
Rarely can the life of a student in an academic community be so private t
hat it will not affect the community as a whole or that the standards above do not apply.
The discovery, advancement, and communication of knowledge 
are not possible without a commitment to these standards. 
Creativity cannot exist without acknowledgment of the creativity of others. 
New knowledge cannot be developed without credit for prior knowledge. 
Without the ability to trust that these principles will be observed, an academic community cannot exist.
The commitment of its faculty, staff and students to these standards contributes to the high respect in which the CMU degree is held. 
Students must not destroy that respect by their failure to meet these standards. 
Students who cannot meet them should voluntarily withdraw from this course.

%%%%%%%%%%%%%%%%%%%%%%%%%%%%%%%%%%%%
\subsection{Cheating and Plagiarism}
%%%%%%%%%%%%%%%%%%%%%%%%%%%%%%%%%%%%
Students in at CMU are engaged in preparation for professional activity of the highest standards. 
Each profession constrains its members with both ethical responsibilities and disciplinary limits. 
To assure the validity of the learning experience a university establishes clear standards for student work. 
In order to deter and detect plagiarism, online tools and other resources are used in this class.
In any presentation, creative, artistic, or research, 
it is the ethical responsibility of each student to identify the conceptual sources of the work submitted. 
Failure to do so is dishonest and is the basis for a charge of cheating or plagiarism, 
which is subject to disciplinary action.

\textbf{Cheating} includes but is not necessarily limited to:
\begin{enumerate}
\item Plagiarism, explained below.
\item Submission of work that is not the student's own for papers, assignments, or exams.
\item Submission or use of falsified data.
\item Theft of or unauthorized access to an exam.
\item Use of an alternate, stand-in or proxy during an examination.
\item Use of unauthorized material including textbooks, notes or computer programs in the preparation of an assignment or during an examination.
\item Supplying or communicating in any way unauthorized information to another student for the preparation of an assignment or during an examination.
\item Collaboration in the preparation of an assignment. Unless specifically permitted or required by the instructor, collaboration will usually be viewed by the university as cheating. Each student, therefore, is responsible for understanding the policies of the department offering any course as they refer to the amount of help and collaboration permitted in preparation of assignments.
\item Submission of the same work for credit in two courses without obtaining the permission of the instructors beforehand.
\end{enumerate}

\textbf{Plagiarism} includes, but is not limited to, failure to indicate the source with quotation marks or footnotes where appropriate if any of the following are reproduced in the work submitted by a student:
\begin{enumerate}
\item A phrase, written or musical.
\item A graphic element.
\item A proof.
\item Specific language.
\item An idea derived from the work, published or unpublished, of another person.
\end{enumerate}
Any disciplinary actions regarding charges of cheating or plagiarism will follow the procedures of the home university of the student involved.
 
\paragraph{Collaboration vs. Cheating}
Collaboration is defined by Merriam-Webster’s Collegiate Dictionary (10th edition) as 
“to work jointly with others or together, especially in an intellectual endeavor.” 
Much of the work that is performed in this laboratory (and in biomedical research as a whole) is collaborative in nature. 
Therefore, collaboration in this class is encouraged during the execution of the labs. 
In addition, discussion regarding the content of homework assignments is also encouraged.
 
You are encouraged to discuss the course material, concepts, and assignments with other students in the class. 
\textbf{However, each student must eventually submit his/her own unique work (e.g., laboratory notebook, final report). }
If any collaboration was used to complete an assignment, 
record the names of the collaborators and the nature of the collaboration. 
Any attempt to submit work that is not the student’s own work will be considered an act of cheating. 
In addition, any student who knowingly supplies their homework assignment for review 
to another student is violating the cheating policy and will be subject to disciplinary action.
 
 \begin{tcolorbox}[colback=red!5,colframe=red!75!black]
ANY VIOLATION OF THIS POLICY WILL NOT BE TOLERATED AND THE PENALTY WILL BE FAILURE IN THE COURSE.
\end{tcolorbox}
 
\textbf{\textit{If you have any questions regarding this policy, please contact the instructor.}}

