\documentclass[11pt, oneside]{article}   	% use "amsart" instead of "article" for AMSLaTeX format

\usepackage{geometry}                		% See geometry.pdf to learn the layout options. There are lots.
\geometry{letterpaper}                   		% ... or a4paper or a5paper or ... 
%\geometry{landscape}                		% Activate for rotated page geometry
\usepackage[parfill]{parskip}    		% Activate to begin paragraphs with an empty line rather than an indent
\usepackage{graphicx}				% Use pdf, png, jpg, or eps§ with pdflatex; use eps in DVI mode
								% TeX will automatically convert eps --> pdf in pdflatex		
\usepackage{amssymb}
\usepackage{amsmath}
\usepackage{tablefootnote}
\usepackage{multirow}
\usepackage{tabularx}
\renewcommand\tabularxcolumn[1]{m{#1}}

\usepackage{dsfont}
\usepackage[hang,flushmargin]{footmisc}

\usepackage{pgfplots,pgfplotstable}

\makeatletter
\long\def\ifnodedefined#1#2#3{%
    \@ifundefined{pgf@sh@ns@#1}{#3}{#2}%
}

\pgfplotsset{
    discontinuous/.style={
    scatter,
    scatter/@pre marker code/.code={
        \ifnodedefined{marker}{
            \pgfpointdiff{\pgfpointanchor{marker}{center}}%
             {\pgfpoint{0}{0}}%
             \ifdim\pgf@y>0pt
                \tikzset{options/.style={mark=*, fill=white}}
                %\draw [densely dashed] (marker-|0,0) -- (0,0);
                \draw plot [mark=*] coordinates {(marker-|0,0)};
             \else
                \tikzset{options/.style={mark=none}}
             \fi
        }{
            \tikzset{options/.style={mark=none}}        
        }
        \coordinate (marker) at (0,0);
        \begin{scope}[options]
    },
    scatter/@post marker code/.code={\end{scope}}
    }
}

\newcommand*\Eval[3]{\left(#1\middle)\right\rvert_{#2}^{#3}}


\usepackage{arydshln}
\usepackage{mathtools}

%Accessibility issues
\usepackage[a-1b]{pdfx}

%\usepackage{tagpdf}
%\tagpdfsetup{activate,paratagging,interwordspace}
\usepackage{accsupp}
\usepackage{hyperref}
\usepackage{lipsum}
%\usepackage{axessibility}
\usepackage{accessibility}

%SetFonts
\DeclareEmphSequence{\bfseries\itshape}
\DeclareMathOperator {\ints}{\mathbb{Z}}
\DeclareMathOperator {\reals}{\mathbb{R}}
\DeclareMathOperator*{\argmin}{argmin}
\DeclareMathOperator*{\argmax}{argmax}


\renewcommand\labelenumi{(\theenumi)}
\renewcommand*{\thefootnote}{\fnsymbol{footnote}}

\usepackage{tcolorbox}

\newenvironment{aside}[1][Aside]{\begin{tcolorbox}[colback=black!5,colframe=black!75!black,title=#1]}{\end{tcolorbox}}

%SetFonts


\title{Topic 11: Calculus Review}
\author{02-680: Essentials of Mathematics and Statistics}
%\date{}							% Activate to display a given date or no date

\begin{document}
\maketitle

%%%%%%%%%%%
\section{Derivatives}
 From some function $f(x)$ where $x$ is a scalar:
 \begin{itemize}
 \item the derivative $\displaystyle\frac{df}{dx}$ is th change in value of $f$ as you increase/decrease $x$.
 \item Formally $\displaystyle\frac{df}{dx}=\lim_{h\rightarrow 0}\frac{f\left(x+h\right) - f(x)}{h}$.
 \end{itemize}
 
 Lets derive one of our known rules below, for instance let $f(x)=x^n$. 
 \begin{align}
 \frac{d(x^n)}{dx}=&\lim_{h\rightarrow 0}\frac{(x-h)^n - x^n}{h}\\
 =& \lim_{h\rightarrow 0}\frac{\sum_{i=0}^n {n \choose i} x^{(n-i)}h^i - x^n}{h}\\
 =& \lim_{h\rightarrow 0}\frac{\sum_{i=1}^{n} {n \choose i} x^{(n-i)}h^i}{h}\\
 =& \lim_{h\rightarrow 0}\sum_{i=1}^{n} {n \choose i} x^{(n-i)}h^{i-1}\\
 =& \lim_{h\rightarrow 0} \left[{n \choose 1}x^{(n-1)}h^{1-1} + \sum_{i=2}^{n} {n \choose i} x^{(n-i)}h^{i-1}\right]\\
 =& nx^{(n-1)} + \lim_{h\rightarrow 0} \left[\sum_{i=2}^{n} {n \choose i} x^{(n-i)}h^{i-1}\right]\\
 =& nx^{(n-1)}
 \end{align}
 

%%%%%%%%%%%
\section{Integrals}
For some function $f'(x)$ where $x$ is a scalar, 
the integral $\displaystyle\int f'(x)$ can be thought of as the inverse of differentiation

For more review, we recommend: 3blue1brown.com/topics/calculus


%%%%%%%%%%%
%\section{Differential Equations}

%%%%%%%%%%%%%%%%%%
\section{Application: Gradient Based Optimization}
\emph{Optimization} is the task of either minimizing or maximizing some function. 
For some $f(x)$, find the $x$ that maximizes $f(x)$. 
Using mathematics: 
\[ x^* = \argmax_{\forall x} f(x) \]
(Note that above we use maximization, but this would be equivalent to minimizing some function $g(x)=-f(x)$.)

Remember the derivative of a function is the rate of change, or slope; 
thus it can be used to tell us how to change $x$ in order to have a maximizing impact on $f(x)$. 

\[\frac{df}{dx} \approx \frac{f(x+\varepsilon)-f(x)}{\varepsilon}\]
we can rewrite $\frac{df}{dx}$ as $f'(x)$ for ease, so then reducing
\[\varepsilon f'(x) \approx f(x+\varepsilon)-f(x)\]
\[f(x) + \varepsilon f'(x) \approx f(x+\varepsilon)\]

Notice that when $f'(x)=0$, we can change x, but nothing happens. 
We call this a \emph{critical} or \emph{stationary} point. 

Lets look at an example, lets say we want to minimize: 
\[f(x) = \frac{x^2}{2}  \;\;\rightarrow\;\; f'(x) = x.\]

If $x>0; f'(x)>0$ and thus to reach a critical point we need to add a negative $\varepsilon$ 
(since we want to minimize the function value). 
Alternatively, if $x<0: f'(x)<0$ and thus we need to add a positive $\varepsilon$. 
If $x=0, f'(x)=0$ and thus we've found a critical point. 

Looking at another example lets say we want to still find the minimum, but for 
\[g(x) = \frac{-x^2}{2}  \;\;\rightarrow\;\; g'(x) = -x.\]
Its still the case that we want to add an $\varepsilon$ opposite the sign of the slope since we want to minimize; 
if $g'(x)<0$ then $\varepsilon>0$, $g'(x)>0$ then $\varepsilon<0$, and $g'(x)=0$ found a critical point. 
(That is the new $x$ should be {\color{red}$x-\alpha f'(x)$}, where $\alpha$ is some adjustable parameter.)

So once again, we have a critical point at $x=0$, 
but if we look at the two examples in the first if we were slightly away from 0 the derivative would have sent us \textit{toward} 0, in the second it sends us away. 
So we can say the critical point of $x=0$ is \emph{stable} for $f(x)$ and \emph{unstable} for $g(x)$. 

What we've described here is a slight simplification of \emph{gradient descent} first described by Cauchy in the 1867. 
One of the most commonly used optimization procedures in machine learning. 

\subsection{Example}
Consider the equation \[3x^4-10x^3-12x^2+18x-3.\]
The derivative is \[12x^3-30x^2-24x+18=6(x-3)(x+1)(2x-1).\]
This has 3 values that are zero: $x=3,-1,\frac{1}{2}.$


To find out if they are local minima or maxima we can take the second derivative: 
\[36x^2-60x-24\] 
and determine if its positive or negative at each point: 
$x=3 (+),-1 (+),\frac{1}{2}(-).$

Thus only $x=3$ and $1$ are candidates for minima, 
we can get their original values:
\[3(3)^4-10(3)^3-12(3)^2+18(3)-3=-84\]
\[3(-1)^4-10(-1)^3-12(-1)^2+18(-1)-3=-20\]

Thus the global minimum is at $x=3$.
%%%%%%%%%%%%%%%%%%
\section{Gradients (Multivariable Derivation)}
When a function has multiple variables we cannot simply take the derivative of the whole thing. 
For instance, lets say we have a function $f:\reals^n\mapsto\reals$, 
as a real example think of euclidean norm: 
\[e(x) = \sqrt{x_1^2+x_2^2}.\]

Now, since $x$ is a vector, when need to take the gradient $\nabla f$ (sort of $\frac{df}{dx}$ when $x$ is a vector). 
We define the gradient as follows:
\[\nabla f = \begin{bmatrix}\frac{\partial f}{\partial x_1}&\frac{\partial f}{\partial x_2}&....&\frac{\partial f}{\partial x_n}\end{bmatrix}\]
when $x\in\reals^n$. 
In the example above that means 
\[\nabla e = \begin{bmatrix}\frac{\partial e}{\partial x_1}&\frac{\partial e}{\partial x_1}\end{bmatrix}= \begin{bmatrix}\frac{x_1}{ \sqrt{x_1^2+x_2^2}}&\frac{x_2}{\sqrt{x_1^2+x_2^2}}\end{bmatrix}\]
Note we need the chain rule then the polynomial rule for both derivatives. 

They key here is to realize that the dimension of the gradient (the number of derivatives) 
is dependent on the dimension of the input to the function (not the output).


%%%%%%%%%%%%%%%%%%
\section{Application: Least Squares Minimization}
Assume we have a problem which can be set up as 
\[Ax=b\]
where $A\in\reals^{m\times n}$, and $b\in\reals^n$.
But let it be the case that $b\notin span(columns(A))$,
therefore we know $\not\exists x\in\reals^n$ such that $Ax=b$.

Can we find something ``close enough''?

Lets say we want to minimize $\|Ax-b\|_2^2$ (we sometimes call this the squared error). 
\begin{aside}[Norm Squared]
Its helpful here to remember that for any vector \[x\in\reals^n: \|x\|_2^2 = \sum_{i=1}^n x_i^2 = x^Tx.\]

The derivation is left as an exercise.
\end{aside}

Remember from above, to find a minimum we need to find the critical points and evaluate them, so we need to find the gradient. 
\[\begin{array}{ccc}
\|Ax-b\|_2^2 & = & (Ax-b)^T(Ax-b)\\
& = & (x^TA^T-b^T)(Ax-b)\\
& = & x^TA^TAx-2AXb^t+b^Tb
\end{array}\]

The gradient with respect to $x$ of this is then $2A^TAx-2A^Tb$.

For this to be 0, \[A^TAx=A^Tb\]
solving for $x$ (assuming $A^TA$ is invertible)
\[x = (A^TA)^{-1}A^Tb.\]


%%%%%%%%%%%%%%%%%%
\section{Jacobian (Multivariable/Multifunction Derivation)}
Sometimes we also have functions that have both multiple variable input, and multiple variable output. 
So any function $g: \reals^n\mapsto\reals^m$ with $m,n\in\reals^{\ge2}$.
We say the \emph{Jacobian} of $g$ is 
\[J_g(x) = \begin{bmatrix}
\frac{\partial f_1}{\partial x_1} & \frac{\partial f_1}{\partial x_2} & \hdots & \frac{\partial f_1}{\partial x_n} \\  
\frac{\partial f_2}{\partial x_1} & \frac{\partial f_2}{\partial x_2} & \hdots & \frac{\partial f_2}{\partial x_n} \\  
\vdots & \vdots & \ddots & \vdots\\
\frac{\partial f_m}{\partial x_1} & \frac{\partial f_m}{\partial x_2} & \hdots & \frac{\partial f_m}{\partial x_n} \end{bmatrix}\]

So for instance $h:\reals^2\mapsto\reals^2$:
\[h(x) = x^T\begin{bmatrix}3 & 1\\0 & -1\end{bmatrix}{\color{red}\begin{bmatrix}x_1 & 0\\ 0 & x_2\end{bmatrix}} 
%{\color{red}=x^T \begin{bmatrix}3*x_1+1*0 & 3*0+1*x_2\\0*x_1+-1*0&0*0+-1*x_2\end{bmatrix}}
{\color{red}=x^T \begin{bmatrix}3x_1 & x_2\\0&-x_2\end{bmatrix}}
%{\color{red}=\begin{bmatrix}3*x_1^2 \\ x_1x_2-x_2^2\end{bmatrix}}
= \begin{bmatrix}3x_1^2\\x_1x_2 -x_2^2\end{bmatrix}\]
then the jacobian is 
\[J_h(x) = \begin{bmatrix} 
\frac{d(3x_1^2)}{x_1} & \frac{d(3x_1^2)}{x_2}\\
\frac{d(x_1x_2 -x_2^2)}{x_1} & \frac{d(x_1x_2 -x_2^2)}{x_2}\end{bmatrix} = 
\begin{bmatrix} 
6x_1 & 0\\
x_2 & x_1-2x_2
\end{bmatrix}
\]


\iffalse
%%%%%%%%%%%%%%%%%%
\section{Application: tRNA production}
Assume all tRNA are divided into two classes: bound (which we will represent the count of at time $t$ as $m_1(t)$) 
and unbound ($m_2(t)$) to an amino acid.

We have some other known variables that are important: 
\begin{itemize}
\item[$\alpha$:] rate of tRNA production
\item[$\beta$:] rate of amino acid binding
\item[$\gamma$:]rate of amino acid loss (unbinding)
\item[$\delta$:] rate of tRNA degradation
\end{itemize}

Our goal will be to find the change in tRNA over time. 
Lets set up some basics:
\[\begin{array}{rrrrr}
m_1(t+1) =&\alpha &- \beta m_1(t)& + \gamma m_2(t)& - \delta m_1(t)\\
m_2(t+1) = & &+ \beta m_1(t) &- \gamma m_2(t)& - \delta m_2(t)
\end{array}\]

Lets do a little more algebra for a second (it will make sense in a moment). \\
First lets simplify: 
\[\begin{array}{rrrrr}
m_1(t+1) =&\alpha &+(- \beta-\delta) m_1(t)& + \gamma m_2(t)\\
m_2(t+1) = & &+ \beta m_1(t) & +(-\gamma- \delta) m_2(t)
\end{array}\]

Then we can rewrite this a little:
\[
\begin{bmatrix}m_1(t+1)\\m_2(t+1)\end{bmatrix} = 
\begin{bmatrix} (- \beta-\delta) & \gamma\\ \beta& (-\gamma- \delta)\end{bmatrix}
\begin{bmatrix}m_1(t)\\m_2(t)\end{bmatrix} +
\begin{bmatrix}\alpha\\0\end{bmatrix}
\]
and with the appropriate matrices:
\[m(t+1)=Mm(t)+b\]

But we can also think of this as being: 
\[J_m(t) = Mm+b\]
\[
J_m(t) = 
\begin{bmatrix} (- \beta-\delta) & \gamma\\ \beta& (-\gamma- \delta)\end{bmatrix}
\begin{bmatrix}m_1(t)\\m_2(t)\end{bmatrix} +
\begin{bmatrix}\alpha\\0\end{bmatrix}
\]

If we want to find stasis we can examine the critical points (when $Mm(t)+b=0$). 

This is an example of a problem you might see in the Modeling and Simulation course. 
\fi


\section*{Useful References}
Deisenroth, Faisal, and Ong, ``Mathematics for Machine Learning''. \S 5

%%%%%%%%%%%%%%%%%%
\clearpage

 \begin{table}
 \caption{Differentiation rules}
\footnotesize
 \vspace{1em} 
 \begin{center}
 \begin{tabularx}{0.9\textwidth}{|l|X|}
 \hline
 constant multiple rule & \[\frac{d}{dx}\left(cf\left(x\right)\right) = cf'\left(x\right)\]\\\hline
polynomial rule &\[\frac{d}{dx}\left(x^n\right) = nx^{n-1}\]\\\hline
\multirow{2}{*}{exponent rules} & \[\frac{d}{dx}\left(e^{g(x)}\right) = e^{g(x)}g'(x)\]\\
 & \[\frac{d}{dx}\left(c^{g(x)}\right) = c^{g(x)}g'(x)\ln c\]\\\hline
 log rule & \[\frac{d}{dx}\left(\log _c g\left(x\right)\right) = \frac{g'(x)}{g(x)\ln a}\]\\\hline
 sum rule & \[\frac{d}{dx}\left(f\left(x\right) + g\left(x\right)\right) = f'(x) + g'(x)\]\\\hline
 product rule & \[\frac{d}{dx}\left(f\left(x\right)g\left(x\right)\right) = f'(x)g(x) + f(x)g'(x)\]\\\hline
 quotient rule & \[\frac{d}{dx}\left(\frac{f(x)}{g(x)}\right) = \frac{f'(x)g(x)-f(x)g'(x)}{g(x)^2}\]\\\hline
 chain rule & \[\frac{d}{dx}\left(f\left(g\left(x\right)\right)\right) = f'\left(g\left(x\right)\right)g'(x)\]\\\hline
 \end{tabularx}
 \end{center}
\end{table}
 \begin{table}
 \caption{Integration rules} 
\vspace{1em} \begin{center}
\footnotesize
 \begin{tabularx}{0.9\textwidth}{|l|X|}
 \hline
 constant multiple rule & \[\int cf\left(x\right) dx = c\int f(x) dx\]\\\hline
polynomial rule &\[\int x^n dx = \frac{x^{n+1}}{a+1} + c\]\\\hline
\multirow{2}{*}{exponent rules} & \[\int e^{x} dx = e^{x}\]\\
 & \[\int c^x dx = \frac{c^x}{\ln c}\]\\\hline
 log rule & \[\int \frac{1}{x} dx = \ln |x| + c\]\\\hline
 sum rule & \[\int\left(f(x)\pm g(x)\right) dx  = \int f(x) dx \pm \int g(x) dx\]\\\hline
 \end{tabularx}
 \end{center}
\end{table}

\end{document}
