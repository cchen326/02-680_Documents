
\documentclass[11pt, oneside]{article}   	% use "amsart" instead of "article" for AMSLaTeX format

\usepackage{geometry}                		% See geometry.pdf to learn the layout options. There are lots.
\geometry{letterpaper}                   		% ... or a4paper or a5paper or ... 
%\geometry{landscape}                		% Activate for rotated page geometry
\usepackage[parfill]{parskip}    		% Activate to begin paragraphs with an empty line rather than an indent
\usepackage{graphicx}				% Use pdf, png, jpg, or eps§ with pdflatex; use eps in DVI mode
								% TeX will automatically convert eps --> pdf in pdflatex		
\usepackage{amssymb}
\usepackage{amsmath}
\usepackage{tablefootnote}
\usepackage{multirow}
\usepackage{tabularx}
\renewcommand\tabularxcolumn[1]{m{#1}}

\usepackage{dsfont}
\usepackage[hang,flushmargin]{footmisc}

\usepackage{pgfplots,pgfplotstable}

\makeatletter
\long\def\ifnodedefined#1#2#3{%
    \@ifundefined{pgf@sh@ns@#1}{#3}{#2}%
}

\pgfplotsset{
    discontinuous/.style={
    scatter,
    scatter/@pre marker code/.code={
        \ifnodedefined{marker}{
            \pgfpointdiff{\pgfpointanchor{marker}{center}}%
             {\pgfpoint{0}{0}}%
             \ifdim\pgf@y>0pt
                \tikzset{options/.style={mark=*, fill=white}}
                %\draw [densely dashed] (marker-|0,0) -- (0,0);
                \draw plot [mark=*] coordinates {(marker-|0,0)};
             \else
                \tikzset{options/.style={mark=none}}
             \fi
        }{
            \tikzset{options/.style={mark=none}}        
        }
        \coordinate (marker) at (0,0);
        \begin{scope}[options]
    },
    scatter/@post marker code/.code={\end{scope}}
    }
}

\newcommand*\Eval[3]{\left(#1\middle)\right\rvert_{#2}^{#3}}


\usepackage{arydshln}
\usepackage{mathtools}

%Accessibility issues
\usepackage[a-1b]{pdfx}

%\usepackage{tagpdf}
%\tagpdfsetup{activate,paratagging,interwordspace}
\usepackage{accsupp}
\usepackage{hyperref}
\usepackage{lipsum}
%\usepackage{axessibility}
\usepackage{accessibility}

%SetFonts
\DeclareEmphSequence{\bfseries\itshape}
\DeclareMathOperator {\ints}{\mathbb{Z}}
\DeclareMathOperator {\reals}{\mathbb{R}}
\DeclareMathOperator*{\argmin}{argmin}
\DeclareMathOperator*{\argmax}{argmax}


\renewcommand\labelenumi{(\theenumi)}
\renewcommand*{\thefootnote}{\fnsymbol{footnote}}

\usepackage{tcolorbox}

\newenvironment{aside}[1][Aside]{\begin{tcolorbox}[colback=black!5,colframe=black!75!black,title=#1]}{\end{tcolorbox}}


\title{Topic 16: Graphical Representations of Random Variables}
\author{02-680: Essentials of Mathematics and Statistics}
%\date{}							% Activate to display a given date or no date

\begin{document}
\maketitle

%%%%%%%%%%%%%%
\section{Bayesian Networks}

Sometimes we need a more complicated conditional description. 
One way to do this is using a network (graph). 

\begin{center}
\begin{dot2tex}[neato,mathmode]
digraph G {
forcelabels=true;
rankdir="LR";
edge [style="-triangle 90"]
node [shape="oval"];
X -- Y; 
X -- Z [constraint=false];
Y -- Z;
}
\end{dot2tex}
\end{center}

In general for a network such as this we define the probabilities as: 
\[p\Big(\left\langle Q_1, Q_2, ..., Q_d\right\rangle\Big) = \prod_{i=0}^d\; p\!\left(Q_i \middle| \bigwedge_{j \in N^{in}(Q_i)} Q_j\right)\]
remember that $N^{in}(v)$ is the set of in-neighbors of $v$ in a graph. 

So for the Bayesian Network above, we may have something like:  
\begin{center}
\begin{tabular}{|c|c|}
\hline
$x$	& $p(X=x)$\\
\hline \hline
0 	& $0.7$\\
1	& $0.3$\\
\hline
\end{tabular}
%
\begin{tabular}{|c|c|c|}
\hline
& \multicolumn{2}{c|}{$p(Y\mid X)$}\\
$x$ & $y=1$	&	$y=0$ \\
\hline \hline
0 	& $0.5$	& $0.5$\\
1	& $0.9$	& $0.1$\\\hline
\end{tabular}
%
\begin{tabular}{|c|c|c|c|}
\hline
&&  \multicolumn{2}{c|}{$p(Z \mid Y, X)$}\\
$x$ & $y$	& $z=1$ & $z=0$\\
\hline \hline
0	&	0 	&	$0.3$	& $0.7$\\
0	&	1	& 	$0.1$	& $0.9$\\
\hline
1	&	0 	&	$0.7$	& $0.3$\\
1	&	1	& 	$0.4$	& $0.6$\\
\hline
\end{tabular}
\end{center}

But if we alter the network to the following (i.e. remove the dependence of $Z$ on $X$), 
\begin{center}
\begin{dot2tex}[neato,mathmode]
digraph G {
forcelabels=true;
rankdir="LR";
edge [style="-triangle 90"]
node [shape="oval"];
X -- Y; 
Y -- Z;
}
\end{dot2tex}
\end{center}
we end up with
\begin{center}
\begin{tabular}{|c|c|}
\hline
$x$	& $p(X=x)$\\
\hline \hline
0 	& $0.7$\\
1	& $0.3$\\
\hline
\end{tabular}
%
\begin{tabular}{|c|c|c|}
\hline
& \multicolumn{2}{c|}{$p(Y\mid X)$}\\
$x$ & $y=1$	&	$y=0$ \\
\hline \hline
0 	& $0.5$	& $0.5$\\
1	& $0.9$	& $0.1$\\\hline
\end{tabular}
%
\begin{tabular}{|c|c|c|}
\hline
&  \multicolumn{2}{c|}{$p(Z \mid Y, X)$}\\
 $y$	& $z=1$ & $z=0$\\
\hline \hline
0 	&	$0.2$	& $0.8$\\
1	& 	$0.7$	& $0.3$\\
\hline
\end{tabular}
\end{center}

%%%%%%%%%%%%%%%
\section{Markov Chains}
Markov chains are usually used to model sequences of items. 
We saw when talking about conditional probabilities that: 
\begin{align}
p(X_n,X_{n-1},...X_1) 	&= p(X_n \mid X_{n-1},...X_1) p(X_{n-1},...X_1)\\
					&= p(X_n \mid X_{n-1},...X_1) p(X_{n-1}\mid X_{n-2},...X_1) p(X_{n-2},...X_1)\\
					& \hdots\\
					& = p(X_n \mid X_{n-1},...X_1) \hdots p(X_2\mid X_1) p(X_1)\\
\end{align}

The \emph{Markov assumption} is that the probability of $X_k$ depends on $X_{k-1}$ alone. 
This simplifies the computation, but at the cost of long term connections. 
Anecdotally the Markov assumption says: 
\begin{center}
the future is independent of the past given the present. 
\end{center}
So that means we can rewrite the statement above as: 
\[p(X_n,X_{n-1},...X_1) = p(X_n \mid X_{n-1}) p(X_{n-1} \mid X_{n-2}) \hdots p(X_2 \mid X_1) p(X_1).\]

This is the basis for \emph{Hidden Markov Models} which are used a lot in genetics. 

\paragraph{Example.} 
Consider a sequence of DNA nucleotides modeled as a set of random variables $N_1N_2\hdots N_n$,
each of which can take on the value $\{\texttt{A},\texttt{T},\texttt{C},\texttt{G}\}$. 
We want to know if the sequence is a CpG island or not. 

Looking at the equation above, if we know the initial distribution (that is the probability distribution for $p(N_1)$) for both cases,
and the transition distributions ($p(N_k\mid N_{k-1})$ we can determine which is more probable. 

\[
P_b = 
\begin{bmatrix}
 & \texttt{A} & \texttt{C} & \texttt{G} & \texttt{T}\\
 \texttt{A} & 0.30 & 0.20& 0.29& 0.21\\
 \texttt{C} & 0.32& 0.30& 0.08& 0.30\\
 \texttt{G} & 0.25 & 0.25 & 0.29 & 0.21\\
 \texttt{T} & 0.18 & 0.24 & 0.29 & 0.29\\
\end{bmatrix} \;\;\;\;
P_c = 
\begin{bmatrix}
 & \texttt{A} & \texttt{C} & \texttt{G} & \texttt{T}\\
 \texttt{A} & 0.18 & 0.27 & 0.43 & 0.12\\ 
 \texttt{C} & 0.17 & 0.37 & 0.27 & 0.19\\
 \texttt{G} & 0.16 & 0.34 & 0.37 & 0.13\\
 \texttt{T} & 0.08 & 0.36 & 0.38 & 0.18\\
\end{bmatrix}
\]

Assuming for both $p(N_1=\texttt{A}) = p(N_1=\texttt{C}) = p(N_1=\texttt{T}) = p(N_1=\texttt{G}) = 0.25$, 
what is the sequence \texttt{ACTTC} more likely to be a CpG island or not?


\section*{Useful References}
Degroot and Schervish. ``Probability and Statistics''  \S\S3.10

\end{document}