\documentclass[11pt, oneside]{article}   	% use "amsart" instead of "article" for AMSLaTeX format

\usepackage{geometry}                		% See geometry.pdf to learn the layout options. There are lots.
\geometry{letterpaper}                   		% ... or a4paper or a5paper or ... 
%\geometry{landscape}                		% Activate for rotated page geometry
\usepackage[parfill]{parskip}    		% Activate to begin paragraphs with an empty line rather than an indent
\usepackage{graphicx}				% Use pdf, png, jpg, or eps§ with pdflatex; use eps in DVI mode
								% TeX will automatically convert eps --> pdf in pdflatex		
\usepackage{amssymb}
\usepackage{amsmath}
\usepackage{tablefootnote}
\usepackage{multirow}
\usepackage{tabularx}
\renewcommand\tabularxcolumn[1]{m{#1}}

\usepackage{dsfont}
\usepackage[hang,flushmargin]{footmisc}

\usepackage{pgfplots,pgfplotstable}

\makeatletter
\long\def\ifnodedefined#1#2#3{%
    \@ifundefined{pgf@sh@ns@#1}{#3}{#2}%
}

\pgfplotsset{
    discontinuous/.style={
    scatter,
    scatter/@pre marker code/.code={
        \ifnodedefined{marker}{
            \pgfpointdiff{\pgfpointanchor{marker}{center}}%
             {\pgfpoint{0}{0}}%
             \ifdim\pgf@y>0pt
                \tikzset{options/.style={mark=*, fill=white}}
                %\draw [densely dashed] (marker-|0,0) -- (0,0);
                \draw plot [mark=*] coordinates {(marker-|0,0)};
             \else
                \tikzset{options/.style={mark=none}}
             \fi
        }{
            \tikzset{options/.style={mark=none}}        
        }
        \coordinate (marker) at (0,0);
        \begin{scope}[options]
    },
    scatter/@post marker code/.code={\end{scope}}
    }
}

\newcommand*\Eval[3]{\left(#1\middle)\right\rvert_{#2}^{#3}}


\usepackage{arydshln}
\usepackage{mathtools}

%Accessibility issues
\usepackage[a-1b]{pdfx}

%\usepackage{tagpdf}
%\tagpdfsetup{activate,paratagging,interwordspace}
\usepackage{accsupp}
\usepackage{hyperref}
\usepackage{lipsum}
%\usepackage{axessibility}
\usepackage{accessibility}

%SetFonts
\DeclareEmphSequence{\bfseries\itshape}
\DeclareMathOperator {\ints}{\mathbb{Z}}
\DeclareMathOperator {\reals}{\mathbb{R}}
\DeclareMathOperator*{\argmin}{argmin}
\DeclareMathOperator*{\argmax}{argmax}


\renewcommand\labelenumi{(\theenumi)}
\renewcommand*{\thefootnote}{\fnsymbol{footnote}}

\usepackage{tcolorbox}

\newenvironment{aside}[1][Aside]{\begin{tcolorbox}[colback=black!5,colframe=black!75!black,title=#1]}{\end{tcolorbox}}

%SetFonts


\title{Topic 12: Introduction to Probability}
\author{02-680: Essentials of Mathematics and Statistics}
%\date{}							% Activate to display a given date or no date

\begin{document}
\maketitle

\begin{center}
Probability is the mathematical language for quantifying uncertainty
\end{center} 

%%%%%%%%%%%%
\section{Basics}
The \emph{sample space} ($\Omega$), sometimes called the universe, is the set of all possible 
\textit{mutually exclusive} outcomes of an event. 
\[\Omega_{coin}=\{\texttt{Heads},\texttt{Tails}\}\]
\[\Omega_{twocoin}=\Omega_{coin}\times\Omega_{coin}=\{
\langle\texttt{Heads},\texttt{Heads}\rangle,
\langle\texttt{Heads},\texttt{Tails}\rangle,
\langle\texttt{Tails},\texttt{Heads}\rangle,
\langle\texttt{Tails},\texttt{Tails}\rangle\}\]

A sample \emph{outcome}, or atomic element, $\omega\in\Omega$ is one of the possible things that can happen. 

A set of events is a subset of the sample space with some condition. 
For instance if the event is that the first of two coin tosses is heads. 
\[A_{firstheads}=\{
\langle\texttt{Heads},\texttt{Heads}\rangle,
\langle\texttt{Heads},\texttt{Tails}\rangle\} \subset \Omega_{twocoin} \]

We say the \emph{probability} $P(A)$ of an event $A$ is the fraction of all outcomes that $A$ covers. 
So in the example above $P(A_{firstheads})=\frac{1}{2}$.

An \emph{event space}, $\mathcal{A}$, is the set of all possible events. 
For discrete events (such as coin flips) this can be thought of typically as the power set of $\Omega$. 
In continuous spaces it is typically thought of as the Borel field of $\Omega$ (details of this are beyond the scope of the class for now). 

Another way of saying this: 
consider non-empty $\Omega$ and $\mathcal{A}$:
\begin{enumerate}
\item $A\in\mathcal{A} \rightarrow \overline{A}\in\mathcal{A}$
\item $A_1,A_2,... \in \mathcal{A} \rightarrow \bigcup_{i=1}^\infty A_i \in \mathcal{A}$.
\end{enumerate}

%%%%%%%%
\subsection{Three Axioms}

\paragraph{Axiom 1: Positive Probability.}
For any event $A$, $p(A)\ge 0$.

That is, we cannot have a negative probability. 
In the previous example if $A_{threetails}$ is the event that you get 3 tails when a coin is tossed twice, 
$p(A_{threetails})=0$ but this can't be negative.

\paragraph{Axiom 2: Total Probability.}
For any event space $\Omega$, $p(\Omega)=1$. 
That is, the probability of something in the sample space happening is 1. 
So for the example above the probability of a single coin flip being either \texttt{Heads} or \texttt{Tails} is 1, 
(as defined) there are no other possible outcomes. 

\paragraph{Axiom 3: Disjoint Event Space.}
For \textit{disjoint} event spaces $A_1, A_2, ..., A_n$, \[p(A_1\cup A_2\cup ... \cup A_n)=p(A_1) + p(A_2) + ... + p(A_n).\]
So in a single coin flip if $A_{heads}$ and $A_{tails}$ are the events of a single coin flip being heads and tails respectively, 
we can see the event spaces are disjoint (they don't share any outcomes)
so \[p(A_{heads} \cup A_{tails}) = p(A_{heads}) + p(A_{tails}) = \frac{1}{2} + \frac{1}{2} = 1.\]

As a counter example lets define \[A_{lastheads}=\{
\langle\texttt{Heads},\texttt{Heads}\rangle,
\langle\texttt{Tails},\texttt{Heads}\rangle\} \subset \Omega_{twocoin}. \]
We can see that 
\[p(A_{firstheads} \cup A_{lastheads}) \ne p(A_{firstheads}) + p(A_{lastheads})\]
because $A_{firstheads} \cap A_{lastheads} = \{\langle\texttt{Heads},\texttt{Heads}\rangle\} \ne \emptyset$ thus they are not disjoint.


%%%%%%%%
\subsection{Example Consequences of the Three Axioms}

\subsubsection{$p(\emptyset) = 0$}
$\Omega \cup \emptyset = \Omega$. 

Since $\Omega$ is disjoint from $\emptyset$, 
we know $p(\Omega\cup\emptyset) = p(\Omega) + p(\emptyset)$. 

Put these two together we see that 
$p(\Omega) = p(\Omega) + p(\emptyset)$. 

Since $p(\Omega)=1$, 
then $1 = 1 + p(\emptyset)$, 
and thus $p(\emptyset) = 0$. 

\subsubsection{$p(\overline{A}) = 1-p(A)$}
We know by definition that $\overline{A} = \Omega \setminus A$. 

From Axiom 3, since $A$ and $\overline{A}$ are disjoint, 
we know $p(A\cup\overline{A}) = p(A) + p(\overline{A})$.

But we also know that $A\cup\overline{A} = \Omega$. 

So that means $ p(A) + p(\overline{A}) = p(\Omega) = 1$. 
With some algebra we find the original statement. 

\subsubsection{$0 \le p(A) \le 1$.}

From axiom 1: $0 \le p(A)$. 

We also know $p(\overline{A})\ge 0$ from axiom 1. (or really  $-1 p(\overline{A})\le 0$)

From above we know $p(A) = 1-p(\overline{A}) \le 1-0 = 1$.

\subsubsection{If $A\subseteq B$ then $p(A) \le p(B)$.}

\begin{align*}
p(B) & = p(A \cup (B\setminus A)) &			B = A \cup (B\setminus A)\\
	&= p(A) + p(B\setminus A) & A \textnormal{ and } B\setminus A \textnormal{ are disjoint (Axiom 3)} \\
	&\ge p(A) 				& p(B\setminus A) \ge 0 \textnormal{ (Axiom 1)} \\
\end{align*}

%%%%%%%%%%%%%
\section{Perspectives}

For a coin toss effect, there are two methods for explaining $p(\texttt{Heads})=\frac{1}{2}$:

\paragraph{Frequentists.} 
You flip a coin 100 times, you get \texttt{Heads} approximately 50 of those tosses. 
Frequentist are very objective in the explanation of things. 

\paragraph{Bayesian.}
You \textit{believe} you will get tails 50\% of the times you flip the coin. 
Bayesian logic is subject to its interpretation. 

\end{document}
