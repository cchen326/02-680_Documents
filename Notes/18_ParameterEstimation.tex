\documentclass[11pt, oneside]{article}   	% use "amsart" instead of "article" for AMSLaTeX format

\usepackage{geometry}                		% See geometry.pdf to learn the layout options. There are lots.
\geometry{letterpaper}                   		% ... or a4paper or a5paper or ... 
%\geometry{landscape}                		% Activate for rotated page geometry
\usepackage[parfill]{parskip}    		% Activate to begin paragraphs with an empty line rather than an indent
\usepackage{graphicx}				% Use pdf, png, jpg, or eps§ with pdflatex; use eps in DVI mode
								% TeX will automatically convert eps --> pdf in pdflatex		
\usepackage{amssymb}
\usepackage{amsmath}
\usepackage{tablefootnote}
\usepackage{multirow}
\usepackage{tabularx}
\renewcommand\tabularxcolumn[1]{m{#1}}

\usepackage{dsfont}
\usepackage[hang,flushmargin]{footmisc}

\usepackage{pgfplots,pgfplotstable}

\makeatletter
\long\def\ifnodedefined#1#2#3{%
    \@ifundefined{pgf@sh@ns@#1}{#3}{#2}%
}

\pgfplotsset{
    discontinuous/.style={
    scatter,
    scatter/@pre marker code/.code={
        \ifnodedefined{marker}{
            \pgfpointdiff{\pgfpointanchor{marker}{center}}%
             {\pgfpoint{0}{0}}%
             \ifdim\pgf@y>0pt
                \tikzset{options/.style={mark=*, fill=white}}
                %\draw [densely dashed] (marker-|0,0) -- (0,0);
                \draw plot [mark=*] coordinates {(marker-|0,0)};
             \else
                \tikzset{options/.style={mark=none}}
             \fi
        }{
            \tikzset{options/.style={mark=none}}        
        }
        \coordinate (marker) at (0,0);
        \begin{scope}[options]
    },
    scatter/@post marker code/.code={\end{scope}}
    }
}

\newcommand*\Eval[3]{\left(#1\middle)\right\rvert_{#2}^{#3}}


\usepackage{arydshln}
\usepackage{mathtools}

%Accessibility issues
\usepackage[a-1b]{pdfx}

%\usepackage{tagpdf}
%\tagpdfsetup{activate,paratagging,interwordspace}
\usepackage{accsupp}
\usepackage{hyperref}
\usepackage{lipsum}
%\usepackage{axessibility}
\usepackage{accessibility}

%SetFonts
\DeclareEmphSequence{\bfseries\itshape}
\DeclareMathOperator {\ints}{\mathbb{Z}}
\DeclareMathOperator {\reals}{\mathbb{R}}
\DeclareMathOperator*{\argmin}{argmin}
\DeclareMathOperator*{\argmax}{argmax}


\renewcommand\labelenumi{(\theenumi)}
\renewcommand*{\thefootnote}{\fnsymbol{footnote}}

\usepackage{tcolorbox}

\newenvironment{aside}[1][Aside]{\begin{tcolorbox}[colback=black!5,colframe=black!75!black,title=#1]}{\end{tcolorbox}}


\title{Topic 18: Parameter Estimation Basics}
\author{02-680: Essentials of Mathematics and Statistics}
%\date{}							% Activate to display a given date or no date

\begin{document}
\maketitle


\emph{What is a ``Statistic''?}
\begin{center}
Definition: Anything that can be computed from the collected data\\ (i.e., must be observable).
\end{center}

Statistics deals with data.

Goal: Make inferences based on data

This process can be divided into three overlapping phases
\begin{enumerate}
\item \textbf{Collecting data} --- collect data in experiments; Preceded by forming hypotheses about phenomena of interest 
\item \textbf{Describing data} --- describe the results
\item \textbf{Analyzing data} --- infer from the results the strength of the evidence with respect to the hypotheses
\end{enumerate}

Generally two types of statistics
\begin{itemize}
\item[] \textbf{Point statistic} --- a single value computed from data (e.g., $\overline{X_n}$)
\item[] \textbf{Interval or range statistics} ---  an interval $a\le x \le b$ computed from the data
\end{itemize} 

Note that a statistic is itself a random variable 
because a new experiment will produce new data to compute it.

%%%%%%%%%%%%%%%
\section{Some Basics}

Consider a dataset $X_1, X_2, \cdots, X_n$ of i.i.d. (independent and identically distributed) random variables from the same unknown distribution. 
That is, for each $X_i$s, the underlying distributions have the same $\mu$ and $\sigma$. 

Let $\overline{X_n}$ be the average of the actual value of the observations:
\[\overline{X_n} = \frac{X_1 + X_2 + \cdots + X_n}{n} = \frac{1}{n} \sum_{i=1}^n X_i.\]
Note this is different from the expected value of the underlying distribution $\mu$.
Note also that $\overline{X_n}$ is also a random variable in and of itself. 

\paragraph{Example. }
Consider a dataset $C_1, C_2,\cdots,C_{100}$ of 100 fair coin flips (chosen iid), with the corresponding random variables 
taking on values of 1 for Heads and 0 for Tails. 
Assume 48 of the 100 coins are heads, then $\overline{C_{100}} = \frac{48}{100}$. 


\paragraph{Law of Large Numbers.} The \emph{law of large numbers} says that as $n$ grows, the probability that $\overline{X_n}$ is close to $\mu$ approaches 1. 
\[\lim_{n\rightarrow\infty} p\left(\overline{X_n}=\mu\right) \mapsto 1.\]

\paragraph{Central Limit Theorem (brief introduction).} While similar to the law of large numbers the \emph{Central Limit Theorem} says that $n$ grows, 
the distribution of $\overline{X_n}$ converges to the Gaussian (normal) distribution $N\left(\mu, \frac{\sigma}{\sqrt{n}}\right)$. 
\[\lim_{n\rightarrow\infty} \overline{X_n} \sim N\left(\mu, \frac{\sigma}{\sqrt{n}}\right).\]

These both say something similar and speak to the fact that more data will make the empirical mean will approach the underlying (theoretical) mean. 
But the Central Limit Theorem is more precise, it allows us to approximate the probability that the sample size is not adequate 
(gives some confidence on the probability from the LLN).
We will see much more about this over the next couple weeks. 



\end{document}