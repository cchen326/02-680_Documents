\documentclass[11pt, oneside]{article}   	% use "amsart" instead of "article" for AMSLaTeX format

\usepackage{geometry}                		% See geometry.pdf to learn the layout options. There are lots.
\geometry{letterpaper}                   		% ... or a4paper or a5paper or ... 
%\geometry{landscape}                		% Activate for rotated page geometry
\usepackage[parfill]{parskip}    		% Activate to begin paragraphs with an empty line rather than an indent
\usepackage{graphicx}				% Use pdf, png, jpg, or eps§ with pdflatex; use eps in DVI mode
								% TeX will automatically convert eps --> pdf in pdflatex		
\usepackage{amssymb}
\usepackage{amsmath}
\usepackage{tablefootnote}
\usepackage{multirow}
\usepackage{tabularx}
\renewcommand\tabularxcolumn[1]{m{#1}}

\usepackage{dsfont}
\usepackage[hang,flushmargin]{footmisc}

\usepackage{pgfplots,pgfplotstable}

\makeatletter
\long\def\ifnodedefined#1#2#3{%
    \@ifundefined{pgf@sh@ns@#1}{#3}{#2}%
}

\pgfplotsset{
    discontinuous/.style={
    scatter,
    scatter/@pre marker code/.code={
        \ifnodedefined{marker}{
            \pgfpointdiff{\pgfpointanchor{marker}{center}}%
             {\pgfpoint{0}{0}}%
             \ifdim\pgf@y>0pt
                \tikzset{options/.style={mark=*, fill=white}}
                %\draw [densely dashed] (marker-|0,0) -- (0,0);
                \draw plot [mark=*] coordinates {(marker-|0,0)};
             \else
                \tikzset{options/.style={mark=none}}
             \fi
        }{
            \tikzset{options/.style={mark=none}}        
        }
        \coordinate (marker) at (0,0);
        \begin{scope}[options]
    },
    scatter/@post marker code/.code={\end{scope}}
    }
}

\newcommand*\Eval[3]{\left(#1\middle)\right\rvert_{#2}^{#3}}


\usepackage{arydshln}
\usepackage{mathtools}

%Accessibility issues
\usepackage[a-1b]{pdfx}

%\usepackage{tagpdf}
%\tagpdfsetup{activate,paratagging,interwordspace}
\usepackage{accsupp}
\usepackage{hyperref}
\usepackage{lipsum}
%\usepackage{axessibility}
\usepackage{accessibility}

%SetFonts
\DeclareEmphSequence{\bfseries\itshape}
\DeclareMathOperator {\ints}{\mathbb{Z}}
\DeclareMathOperator {\reals}{\mathbb{R}}
\DeclareMathOperator*{\argmin}{argmin}
\DeclareMathOperator*{\argmax}{argmax}


\renewcommand\labelenumi{(\theenumi)}
\renewcommand*{\thefootnote}{\fnsymbol{footnote}}

\usepackage{tcolorbox}

\newenvironment{aside}[1][Aside]{\begin{tcolorbox}[colback=black!5,colframe=black!75!black,title=#1]}{\end{tcolorbox}}


\title{Topic 20: Hypothesis Testing}
\author{02-680: Essentials of Mathematics and Statistics}
%\date{}							% Activate to display a given date or no date

\begin{document}
\maketitle

Once we have taken our data and tried to fit a model to it, 
we often want to know if that fit model matches our original assumptions about the data. 
We call this \emph{hypothesis testing}, 
and we use these techniques to put an actual value on this match. 

We add some terminology on what we've been talking about with respect to statistics 
and frame things as follows: 
we first define our \emph{alternate hypothesis}, which we will denote $H_1$, 
this will be the set of events that define what we want to ask about the confidence in happening; 
we then define the \emph{null hypothesis}, which we denote $H_0$, 
this is the set of events that are all outcomes other than than our alternate. 
%So if we're asking about a coin being unfair, the alternate would be that its biased and the null hypothesis is that its fair. 
So lets say were asking if a drug has a measurable impact on cholesterol, 
the null hypothesis would be that cholesterol stayed the same 
and the alternate hypothesis would be that it changed. 

We usually refer to a hypothesis test telling us if we should \emph{reject} or \emph{retain} the null hypothesis. 

%%%%%%%%%%%
\section{Defining Errors}
Errors occur when the hypothesis test tells us something thats wrong. 
So in the example above 
if the test tells us to reject the null (that is, its confident that the cholesterol changed) 
but in reality it didn't change we call this a \emph{Type I} error. 
On the other hand if our test tells us to retain the null but in reality the value \textit{did} change 
we call that a \emph{Type II} error. 

\begin{center}
\begin{tabular}{|cc||c|c|}
\hline
&& \multicolumn{2}{c|}{Hypothesis Test Result}\\
&& Retain $H_0$ & Reject $H_0$\\
\hline \hline
\multirow{2}{*}{Truth} & $H_0$ & & Type I Error\\
\cline{2-4}
& $H_1$ & Type II Error & \\
\hline
\end{tabular}
\end{center}

We say the 
\begin{itemize}
\item[] \textbf{Type I Error Rate} is $p(\text{reject }H_0, H_0 \text{ is true})$,  
\item[] \textbf{Type II Error Rate} is $p(\text{retain }H_0, H_1 \text{ is true})$, and 
\item[] Statistical \emph{Power} is 1 - Type II Error Rate. 
\end{itemize}
The last point means that the higher power tests have a stronger ability to detect signals for $H_1$. 

Let's look at it visually, first for what we call a \emph{two-sided} test, 
that is $H_0: \mu = x$ and $H_1: \mu \ne x$. 
\begin{center}
\begin{tikzpicture}[
    declare function={gaus(\k,\m,\s)=1/(\s*sqrt(2*pi))*exp(-((\k-\m)^2)/(2*\s^2));}
]
\begin{axis}[
     width=.8\textwidth,
     height=.3\textwidth,
    xmin=-5,xmax=5,
    xlabel={$\omega$},
    ylabel={$p(\omega)$},
    samples=100,smooth,
    xtick={-2,2},
    xticklabels={$\mu_0$,$\mu^*$},
    ytick={0},yticklabels={},
]
\addplot [blue] {gaus(x,-2,1)}; \addlegendentry{$H_0$}
\addplot [red] {gaus(x,2,1)}; \addlegendentry{$H_0$}
\draw[ultra thin, dashed] (axis cs:-2,0) -- (axis cs:-2,0.4);
\draw[ultra thin, dashed] (axis cs:2,0) -- (axis cs:2,0.4);
\end{axis}
\end{tikzpicture}
\end{center}



\end{document}