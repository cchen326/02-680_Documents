\documentclass[11pt, oneside]{article}   	% use "amsart" instead of "article" for AMSLaTeX format

\usepackage{geometry}                		% See geometry.pdf to learn the layout options. There are lots.
\geometry{letterpaper}                   		% ... or a4paper or a5paper or ... 
%\geometry{landscape}                		% Activate for rotated page geometry
\usepackage[parfill]{parskip}    		% Activate to begin paragraphs with an empty line rather than an indent
\usepackage{graphicx}				% Use pdf, png, jpg, or eps§ with pdflatex; use eps in DVI mode
								% TeX will automatically convert eps --> pdf in pdflatex		
\usepackage{amssymb}
\usepackage{amsmath}
\usepackage{tablefootnote}
\usepackage{multirow}
\usepackage{tabularx}
\renewcommand\tabularxcolumn[1]{m{#1}}

\usepackage{dsfont}
\usepackage[hang,flushmargin]{footmisc}

\usepackage{pgfplots,pgfplotstable}

\makeatletter
\long\def\ifnodedefined#1#2#3{%
    \@ifundefined{pgf@sh@ns@#1}{#3}{#2}%
}

\pgfplotsset{
    discontinuous/.style={
    scatter,
    scatter/@pre marker code/.code={
        \ifnodedefined{marker}{
            \pgfpointdiff{\pgfpointanchor{marker}{center}}%
             {\pgfpoint{0}{0}}%
             \ifdim\pgf@y>0pt
                \tikzset{options/.style={mark=*, fill=white}}
                %\draw [densely dashed] (marker-|0,0) -- (0,0);
                \draw plot [mark=*] coordinates {(marker-|0,0)};
             \else
                \tikzset{options/.style={mark=none}}
             \fi
        }{
            \tikzset{options/.style={mark=none}}        
        }
        \coordinate (marker) at (0,0);
        \begin{scope}[options]
    },
    scatter/@post marker code/.code={\end{scope}}
    }
}

\newcommand*\Eval[3]{\left(#1\middle)\right\rvert_{#2}^{#3}}


\usepackage{arydshln}
\usepackage{mathtools}

%Accessibility issues
\usepackage[a-1b]{pdfx}

%\usepackage{tagpdf}
%\tagpdfsetup{activate,paratagging,interwordspace}
\usepackage{accsupp}
\usepackage{hyperref}
\usepackage{lipsum}
%\usepackage{axessibility}
\usepackage{accessibility}

%SetFonts
\DeclareEmphSequence{\bfseries\itshape}
\DeclareMathOperator {\ints}{\mathbb{Z}}
\DeclareMathOperator {\reals}{\mathbb{R}}
\DeclareMathOperator*{\argmin}{argmin}
\DeclareMathOperator*{\argmax}{argmax}


\renewcommand\labelenumi{(\theenumi)}
\renewcommand*{\thefootnote}{\fnsymbol{footnote}}

\usepackage{tcolorbox}

\newenvironment{aside}[1][Aside]{\begin{tcolorbox}[colback=black!5,colframe=black!75!black,title=#1]}{\end{tcolorbox}}


\title{Review via Linear Regression}
\author{02-680: Essentials of Mathematics and Statistics}
%\date{}							% Activate to display a given date or no date

\begin{document}
\maketitle

The \emph{linear regression} problem is as follows 
(we covered this briefly in Topic 10, but with a slightly different formula):
\begin{itemize}
\item[\textbf{$x$}] --- Independent variable[s] (what we control)
\item[\textbf{$y$}] --- Dependent variable[s] (the measured values)
\item[\textbf{$\beta_s$}] -- slope
\item[\textbf{$\beta_0$}] -- intercept
\end{itemize} 


Some examples of regression in biology:
\begin{center}\begin{tabular}{|c|c|}
\hline
Independent & Dependent\\
\hline\hline
Expression of genes expressed in immune response & asthma severity\\
\hline
Expression of regulator genets & Expression of target genes\\
\hline
Clinical variables & Insulin level\\
\hline
... & ...
\end{tabular}\end{center}

Lets assume we have some $x,y\in\reals^n$, for $n$ samples where each $(x_i,y_i)$ are corresponding. 
Then \[y = x^T\beta_s + \beta_0\]

To simplify things lets define:
$X = \begin{bmatrix} x_1 & 1\\x_2 & 1\\\vdots & \vdots\\x_n & 1\end{bmatrix}$ and $\beta = \begin{bmatrix}\beta_s\\\beta_0\end{bmatrix}$, 
then the equation above becomes simply:
\[y = X\beta.\]

Recall (possibly) from earlier that we want to find $\beta$ that minimizes the $\|y-X\beta\|^2_2$, 
or the square of the $L_2$-norm. 

\begin{align*}
\|y-X\beta\|^2_2 & = (y-X\beta)^T(y-X\beta)\\
& =  (y^T-\beta^TX^T)(y-X\beta)\\
& = y^Ty + \beta^TX^TX\beta - 2 \beta^TX^Ty
\end{align*}

To find the critical point of the curve (i.e. the minimum) we need
\begin{align*}
\frac{d}{d\beta} y^Ty + \beta^TX^TX\beta - 2 \beta^TX^Ty &= 0\\
\frac{d}{d\beta}\beta^TX^TX\beta^2  - \frac{d}{d\beta}2 \beta^TX^Ty &= 0\\
2 X^TX\beta - 2 X^Ty &=0\\
2 X^TX\beta &= 2 X^Ty \\
\beta &= X^Ty(X^TX)^{-1}
\end{align*}

What if I want to know the probability of some $y_i$ given $x_i$? (so $p(y_i\mid x_i)$)

Assuming its some natural process, we can assume 
\[p(y_i\mid x_i) \sim \mathcal{N}(\beta_0 + \beta_s x_i, \sigma^2).\]
or equivalently
\[y= \beta_0 + \beta_s x_i + \epsilon, \epsilon \sim \mathcal{N}(0,\sigma^2).\]

So given the data, we can actually use MLE to find $\beta$!

\begin{align*}
\ln p(\mathcal{D}\mid \beta, \sigma^2) &= \ln \left[\prod_{i=1}^{n} \frac{1}{\sigma^2\sqrt{2\pi}} e^{-\frac{1}{2\sigma^2}(y_i-x_i\beta)^2}\right]\\
	&= \sum_{i=1}^{n} \ln \left[\frac{1}{\sigma^2\sqrt{2\pi}} e^{-\frac{1}{2\sigma^2}(y_i-x_i\beta)^2}\right]\\
	&=  \frac{-n}{2}\ln \sigma^2 + \sum_{i=1}^{n} \ln e^{-\frac{1}{2\sigma^2}(y_i-x_i\beta)^2}\\
	&= \frac{-n}{2}\ln \sigma^2 -\frac{1}{2\sigma^2}\sum_{i=1}^{n} (y_i-x_i\beta)^2
\end{align*}

\begin{align*}
\frac{\partial}{\partial\beta} \frac{-n}{2}\ln \sigma^2 -\frac{1}{2\sigma^2}\sum_{i=1}^{n} (y_i-x_i\beta)^2 &= \frac{\partial}{\partial\beta} -\frac{1}{2\sigma^2}\sum_{i=1}^{n} (y_i-x_i\beta)^2\\
&=  \frac{\partial}{\partial\beta} -\frac{1}{2\sigma^2} (Y-X\beta)^2\\
&= -\frac{1}{2\sigma^2}  \frac{\partial}{\partial\beta} (Y^2-2Y^TX\beta+\beta^TX^TX\beta)\\
&= ... \\
\end{align*}

$\hat\beta_{MLE} = X^Ty(X^TX)^{-1}$, thats just Least Squares!

But we don't know $\sigma$, thats actually, what we want to know because it models how uncertain we are with any given output.

\begin{align*}
\frac{\partial}{\partial\sigma^2} \frac{-n}{2}\ln \sigma^2 -\frac{1}{2\sigma^2}\sum_{i=1}^{n} (y_i-x_i\beta)^2 &= \frac{-n}{2\sigma^2} +\frac{1}{2}(\sigma^2)^{-2}\sum_{i=1}^{n} (y_i-x_i\beta)^2
\end{align*}

$\hat\sigma^2_{MLE} = \frac{\sum_{i=1}^{n} (y_i-x_i\beta)^2}{n}$

With these we can determine for some new $\tilde y$ and $\tilde x$ what $p(\tilde y \mid \tilde x)$ is. 

\end{document}