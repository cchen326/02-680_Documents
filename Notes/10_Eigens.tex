\documentclass[11pt, oneside]{article}   	% use "amsart" instead of "article" for AMSLaTeX format

\usepackage{geometry}                		% See geometry.pdf to learn the layout options. There are lots.
\geometry{letterpaper}                   		% ... or a4paper or a5paper or ... 
%\geometry{landscape}                		% Activate for rotated page geometry
\usepackage[parfill]{parskip}    		% Activate to begin paragraphs with an empty line rather than an indent
\usepackage{graphicx}				% Use pdf, png, jpg, or eps§ with pdflatex; use eps in DVI mode
								% TeX will automatically convert eps --> pdf in pdflatex		
\usepackage{amssymb}
\usepackage{amsmath}
\usepackage{tablefootnote}
\usepackage{multirow}
\usepackage{tabularx}
\renewcommand\tabularxcolumn[1]{m{#1}}

\usepackage{dsfont}
\usepackage[hang,flushmargin]{footmisc}

\usepackage{pgfplots,pgfplotstable}

\makeatletter
\long\def\ifnodedefined#1#2#3{%
    \@ifundefined{pgf@sh@ns@#1}{#3}{#2}%
}

\pgfplotsset{
    discontinuous/.style={
    scatter,
    scatter/@pre marker code/.code={
        \ifnodedefined{marker}{
            \pgfpointdiff{\pgfpointanchor{marker}{center}}%
             {\pgfpoint{0}{0}}%
             \ifdim\pgf@y>0pt
                \tikzset{options/.style={mark=*, fill=white}}
                %\draw [densely dashed] (marker-|0,0) -- (0,0);
                \draw plot [mark=*] coordinates {(marker-|0,0)};
             \else
                \tikzset{options/.style={mark=none}}
             \fi
        }{
            \tikzset{options/.style={mark=none}}        
        }
        \coordinate (marker) at (0,0);
        \begin{scope}[options]
    },
    scatter/@post marker code/.code={\end{scope}}
    }
}

\newcommand*\Eval[3]{\left(#1\middle)\right\rvert_{#2}^{#3}}


\usepackage{arydshln}
\usepackage{mathtools}

%Accessibility issues
\usepackage[a-1b]{pdfx}

%\usepackage{tagpdf}
%\tagpdfsetup{activate,paratagging,interwordspace}
\usepackage{accsupp}
\usepackage{hyperref}
\usepackage{lipsum}
%\usepackage{axessibility}
\usepackage{accessibility}

%SetFonts
\DeclareEmphSequence{\bfseries\itshape}
\DeclareMathOperator {\ints}{\mathbb{Z}}
\DeclareMathOperator {\reals}{\mathbb{R}}
\DeclareMathOperator*{\argmin}{argmin}
\DeclareMathOperator*{\argmax}{argmax}


\renewcommand\labelenumi{(\theenumi)}
\renewcommand*{\thefootnote}{\fnsymbol{footnote}}

\usepackage{tcolorbox}

\newenvironment{aside}[1][Aside]{\begin{tcolorbox}[colback=black!5,colframe=black!75!black,title=#1]}{\end{tcolorbox}}

%SetFonts


\title{Topic 10: Eigenvalues and Eigenvectors}
\author{02-680: Essentials of Mathematics and Statistics}
%\date{}							% Activate to display a given date or no date

\begin{document}
\maketitle

For a square matrix $A\in\reals^{n\times n}$ for which $Ax=\lambda x$, 
we define $\lambda\in\reals$ as an \emph{eigenvalue} and $x\in\reals^{n}\setminus\mathbf{0}$ as an \emph{eigenvector}. 
Together we call these \emph{eigenelements}. 

The main idea is that the \emph{transformation} of $x$ by $A$ results in a vector that points in the same direction of $x$, but scaled by $\lambda$. 

Lets look at an example: let $A=\begin{bmatrix}3 & 4 \\ 2 & 1\end{bmatrix}$, $x=\begin{bmatrix}2\\1\end{bmatrix}$, and $\lambda=5$.
\[\begin{array}{rcl}
Ax&  \stackrel{?}{=}  &\lambda x\\
\begin{bmatrix}3 & 4 \\ 2 & 1\end{bmatrix}\begin{bmatrix}2\\1\end{bmatrix} &  \stackrel{?}{=}  & 5\begin{bmatrix}2\\1\end{bmatrix}\\
\begin{bmatrix}(3\cdot2)+(4\cdot1)\\(2\cdot2)+(1\cdot1)\end{bmatrix} &  \stackrel{?}{=}  & \begin{bmatrix}(5\cdot2)\\(5\cdot1)\end{bmatrix}\\
\begin{bmatrix}10\\5\end{bmatrix} & = & \begin{bmatrix}10\\5\end{bmatrix}\\
\end{array}\]

but given only $A$, how can we find the eigenelements? 

\section{Finding Eigenelements}
Lets break down the original equation: 
\[\begin{array}{rcl}
Ax&=&\lambda x\\
Ax&=&\lambda I_n x\\
Ax-\lambda I_n x&=&\lambda I_n x-\lambda I_n x\\
(A-\lambda I_n)x & = & 0\\
\end{array}\]

Since we know $x$ cannot be $\mathbf{0}$ then $(A-\lambda I_n)$ but be 0 to satisfy the equation. 
This is actually a linear system on one variable!
It turns out this only has non-trivial solutions when $(A-\lambda I_n)$ is singular (that is, it can't be invertible). 

And remember we know that a matrix is singular when the determinant is $0$: 
 \[|A-\lambda I_n|=0.\]
We call this a \emph{characteristic equation}. 

Lets look at the example from above, 
we can rewrite it this way: 
\[\begin{array}{rcl}
\begin{bmatrix}3 & 4 \\ 2 & 1\end{bmatrix}-\lambda\begin{bmatrix}1&0\\0&1\end{bmatrix} & = & 0\\
\begin{bmatrix}3-\lambda & 4 \\ 2 & 1-\lambda\end{bmatrix} & = & 0\\
(3-\lambda)(1-\lambda)-(4)(2) & = & 0\\
(3-4\lambda+\lambda^2)-(8) & = & 0\\
\lambda^2-4\lambda-5 & = & 0\\
(\lambda+1)(\lambda - 5)  & = & 0\\
\end{array}\]

So we know that this matrix is singular when $\lambda = 5$ and $-1$. 
These are the eigenvalues of $A$, to find the eigenvectors we plug this back into the original equation and solve for $x$. 
\[\begin{array}{rcl}
\left(\begin{bmatrix}3 & 4 \\ 2 & 1\end{bmatrix}-5\begin{bmatrix}1&0\\0&1\end{bmatrix}\right)\begin{bmatrix}x_1\\x_2\end{bmatrix} & = & 0\\
\begin{bmatrix}-2 & 4 \\ 2 & -4\end{bmatrix}\begin{bmatrix}x_1\\x_2\end{bmatrix} & = & 0\\
\end{array}\]

We can use whatever method we want to solve the system but we will find that for \[\lambda=5, x=\begin{bmatrix}2\\1\end{bmatrix}.\]
Similarly for $\lambda=-1$ we get the system:  
\[\begin{bmatrix}4 & 4 \\ 2 & 2\end{bmatrix}\begin{bmatrix}x_1\\x_2\end{bmatrix} = 0 \rightarrow x=\begin{bmatrix}1\\-1\end{bmatrix}.\]
Note in both of these cases there is a trivial solution of $x=\mathbf{0}$ but this is disallowed by definition. 

\begin{aside}
Consequence of above are 
\begin{itemize}
\item That the determinant is the sum of the eigenvalues: \[|A|=\sum_{i=1}^n \lambda_i\]
\item The rank of $A$ is equal to the number of non-zero eigenvalues
\item If $A$ is nonsingular, then $1/\lambda_i$ is an eigenvalue of $A^{-1}$ with the same associated original eigenvector
\end{itemize}
\end{aside}

\paragraph{A larger example. }
Find the eigenelements of $B=\begin{bmatrix}1&1&-1\\2&3&-4\\4&1&-4\end{bmatrix}.$

We first write down the characteristic equation:
\[\begin{array}{rcl}
\left|\begin{matrix}1-\lambda&1&-1\\2&3-\lambda&-4\\4&1&-4-\lambda\end{matrix}\right| & = & 0\\
(1)\left|\begin{matrix}2&-4\\4&-4-\lambda\end{matrix}\right| - 
(3-\lambda)\left|\begin{matrix}1-\lambda&-1\\4&-4-\lambda\end{matrix}\right| + 
(1)\left|\begin{matrix}1-\lambda&-1\\2&-4\end{matrix}\right| & = & 0\\
\textnormal{(note we chose column 2 because it had more 1's)}\\
\textnormal{... details left to the reader ... }\\
%((2)(-4-\lambda)-(-4)(4)) - (3-\lambda)((1-\lambda)(-4-\lambda)-(-1)(4)) \\+ ((1-\lambda)(-4)-(-1)(2)) & = & 0\\
%((-8-2\lambda)+16) - (3-\lambda)(-4+4\lambda-\lambda+\lambda^2+4) \\+ (-4+4\lambda+2) & = & 0\\
%(-2\lambda+8) - (3-\lambda)(3\lambda+\lambda^2) + (4\lambda-2) & = & 0\\
%(-2\lambda+8) - (9\lambda-3\lambda^2+3\lambda^2-\lambda^3) + (4\lambda-2) & = & 0\\
%(-2\lambda+8) - (9\lambda-\lambda^3) + (4\lambda-2) & = & 0\\
%\lambda^3 + (-2-9+4)\lambda + (8-2) & = & 0\\
\lambda^3 -7 \lambda + 6 & = & 0\\
(\lambda-1)(\lambda-2)(\lambda+3) & = & 0\\
\end{array}\]
Therefore our eigenvalues are $\lambda=1,2,$ and $-3$. 

If we solve the system: \[\left(\begin{bmatrix}1&1&-1\\2&3&-4\\4&1&-4\end{bmatrix}-\begin{bmatrix}1&0&0\\0&1&0\\0&0&1\end{bmatrix}\right)\begin{bmatrix}x_1\\x_2\\x_3\end{bmatrix}=\begin{bmatrix}0\\0\\0\end{bmatrix}\] we find that $x_1=x_2=x_3.$ So we can say $x=\begin{bmatrix}1\\1\\1\end{bmatrix}$.

Finding the other eigenvectors is left as an exercise. 

\section{Triangular Matrices}
A matrix $T\in\reals^{n\times n}$ is triangular if all values on one side of the diagonal are 0:
\[
\begin{bmatrix}
T_{11} & T_{12} & \hdots & T_{1(n-1)} & T_{1n}\\
0 & T_{22} & \hdots & T_{2(n-1)} & T_{2n}\\
\vdots & \vdots & \ddots & \vdots & \vdots\\
0 & 0 & \hdots & T_{(n-1)(n-1)} & T_{(n-1)n}\\
0 & 0 & \hdots & 0 & T_{nn}\\
\end{bmatrix}
\textnormal{or} 
\begin{bmatrix}
T_{11} & 0 & \hdots & 0 & 0 \\
T_{11} & T_{22} & \hdots & 0 & 0\\
\vdots & \vdots & \ddots & \vdots & \vdots\\
T_{(n-1)1} & T_{(n-1)2} & \hdots & T_{(n-1)(n-1)} &0\\
T_{n1} & T_{n2} & \hdots & T_{n(n-1)} & T_{nn}\\
\end{bmatrix}.
\]

It turns out triangular matrices have some special properties: 
\begin{itemize}
\item $\displaystyle |T| = \sum_{i=1}^n T_{ii}$, and 
\item $\langle \lambda_1, \lambda_2, ..., \lambda_n\rangle = \langle T_{11}, T_{22}, ... T_{nn}\rangle$. 
\end{itemize}

\paragraph{Example.} Find the eigenelements of $D=\begin{bmatrix}0&1&2&1\\0&1&0&2\\0&0&1&0\\0&0&0&2\end{bmatrix}$.

\[\begin{array}{rcl}
\left|\begin{matrix}-\lambda&1&2&1\\0&1-\lambda&0&2\\0&0&1-\lambda&1\\0 & 0 & 0 & 2-\lambda\end{matrix}\right| & = & 0\\
-\lambda\left|\begin{matrix}1-\lambda&0&2\\0&1-\lambda&1\\0 & 0 & 2-\lambda\end{matrix}\right| & = & 0\\
(-\lambda)(1-\lambda)\left|\begin{matrix}1-\lambda&1\\0 & 2-\lambda\end{matrix}\right| & = & 0\\
(-\lambda)(1-\lambda)^2(2-\lambda) & = & 0\\
\end{array}\]

We know $\lambda$s are $0,1,2$ (since they are the distinct values). 

We can solve for each $\lambda$:

$\lambda=0$: 
\[(D-0I_4)x = 0 \rightarrow \left(\begin{array}{rcrcrcr} x_2 & + & 2x_3 & + & x_4 & = & 0\\ x_2 & + &  & + & 2x_4 & = & 0\\ &  & x_3 &&&  = & 0\\ & &&& 2x_4 & = & 0\\\end{array}\right)\rightarrow x = \begin{bmatrix}1\\0\\0\\0\end{bmatrix}\]
Notice that $x_1$ is an arbitrary value. 

$\lambda=2$:
\[(D-2I_4)x = 0 \rightarrow \left(\begin{array}{rcrcrcrcr} -2x_1 & + & x_2 & + & 2x_3 & + & x_4 & = & 0\\ &&-x_2 & + &  & + & 2x_4 & = & 0\\ &&&  & -x_3 &&&  = & 0\\ &&& &&& 0& = & 0\\\end{array}\right)\rightarrow x = \begin{bmatrix}\frac{3}{2}x_4\\2x_4\\0\\x_4\end{bmatrix}=\begin{bmatrix}3\\4\\0\\2\end{bmatrix}\]
Notice that $x_4$ is an arbitrary value, but chosen to make the vector integers.  

$\lambda=1$:
\[(D-I_4)x = 0 \rightarrow \left(\begin{array}{rcrcrcrcr} -x_1 & + & x_2 & + & 2x_3 & + & x_4 & = & 0\\ && & &  &  & 2x_4 & = & 0\\ &&&  & 0 &&&  = & 0\\ &&& &&& x_4 & = & 0\\\end{array}\right)\rightarrow\]
\[\left(\begin{array}{rcrcrcrcr}-x_1 & + & x_2 & + & 2x_3 &  &  & = & 0\\ &&& &&& x_4 & = & 0\\\end{array}\right) \rightarrow x = \begin{bmatrix}1\\1\\0\\0\end{bmatrix} \textnormal{ and }\begin{bmatrix}2\\0\\1\\0\end{bmatrix}\]
(Note these are \textit{bases} of the space defined by the system above.)



\paragraph{Symmetric Matrices.}
If a matrix is symmetric (that is $A=A^T$): 
\begin{itemize}
\item all eigenvalues are real, and 
\item eigenvectors are orthonormal. 
\end{itemize}


\section{Eigendecomposition and Diagonalization}
If we create a new matrix $X\in\reals^{n\times n}$ where each \textit{column} is one of the eigenvectors of $A$, 
then create $\Lambda\in\reals^{n\times n}$ which contains the eigenvalues on the diagonal (i.e. $\Lambda = \langle\lambda_1,\lambda_2,...,\lambda_n\rangle I_n$ it turns out that 
\[ AX = X\Lambda \]
because it satisfies all of the eigenelement sets simultaneously. 

If the eigenvectors are linearly independent then $X$ is invertable. 
If $X$ is invertable, then 
\[ A = X\Lambda X^{-1}.\]

We call a $A$ \emph{diagonalizable} if it can be rewritten this way. 
This is sometimes also called an \emph{eigendecomposition}.

\subsection{Singlar Value Decomposition}
While we're only going to briefly discuss it in this class
the form \[ A = X\Lambda X^{-1}\]
is similar to whats known as the \emph{Singular Value Decomposition} (SVD)
of a rectangular matrix $R\in\reals^{n\times m}$
as 
\[R=USV\]
 (sometimes $\Sigma$ is used in place of $S$ but this is a reserved character in this class) 
 where $U\in\reals^{n\times n}$, $V\in\reals^{m\times m}$, 
 and $S\in\left\{\widehat{diag_{n\times m}}(x) \mid x\in\reals^{min(n,m)}\right\}\subset\reals^{n\times m}$.
 Here the \[\widehat{diag_{n\times m}}\left(\left\langle x_1,x_2,...x_m\right\rangle\right) = 
 \begin{bmatrix}x_1 & 0 & \hdots & 0\\0 & x_2 & \hdots & 0\\\vdots&\vdots&\ddots&\vdots\\0 & 0 & \hdots & x_m\\0 & 0 & \hdots &0\\\vdots&\vdots&\vdots&\vdots\\\end{bmatrix} \textnormal{ or }
 \begin{bmatrix}x_1 & 0 & \hdots & 0 & 0 & \hdots\\0 & x_2 & \hdots & 0 & 0 & \hdots\\\vdots&\vdots&\ddots&\vdots\\0 & 0 & \hdots & x_m & 0 & \hdots\\\end{bmatrix}
 .\] 
 It produces the left matrix when $n>m$, and the right when $m$ is larger; 
 that is it creates a diagonal matrix padded with 0 rows/columns to make it the correct size.
 
 Lets assume you're in the situation where you have $n$ users who reviewed $m$ movies and those ratings are in matrix $E\in\reals^{n\times m}$. 
 If you can find the SVD of the matrix such that $E=FGH$ such that $F\in\reals^{n\times m}$, $H\in\reals^{m\times m}$ and $G$ is a diagonal $m$-dimension square matrix each of these represents something about your set. 
 $F$ and $H$ show commonalities between users or movies respectively, 
 and $G$ is a connection matrix. 
 
 Another example would be gene correlations, assume you have multiple assays of gene's expressions in various conditions. 
 If you can decompose that matrix using the theory above you'd have a correlation matrix of genes, correlation of assays, and a relationship matrix. 
 
 
 The details about finding this are beyond the scope of this course. 


\section*{Useful References}
Isaak and Monougian, ``Basic Concepts of Linear Algebra''. \S 7\\
Wilder, ``10-606-f23:Lecture 6'' GitHub repository, \url{https://github.com/bwilder0/10606-f23/blob/main/files/notes_vectorspace.pdf}\\
Kolter, ``Linear Algebra Review and Reference'', \url{https://www.cs.cmu.edu/~zkolter/course/15-884/linalg-review.pdf} \S 3.12

\end{document}
