\documentclass[11pt, oneside]{article}   	% use "amsart" instead of "article" for AMSLaTeX format

\usepackage{geometry}                		% See geometry.pdf to learn the layout options. There are lots.
\geometry{letterpaper}                   		% ... or a4paper or a5paper or ... 
%\geometry{landscape}                		% Activate for rotated page geometry
\usepackage[parfill]{parskip}    		% Activate to begin paragraphs with an empty line rather than an indent
\usepackage{graphicx}				% Use pdf, png, jpg, or eps§ with pdflatex; use eps in DVI mode
								% TeX will automatically convert eps --> pdf in pdflatex		
\usepackage{amssymb}
\usepackage{amsmath}
\usepackage{tablefootnote}
\usepackage{multirow}
\usepackage{tabularx}
\renewcommand\tabularxcolumn[1]{m{#1}}

\usepackage{dsfont}
\usepackage[hang,flushmargin]{footmisc}

\usepackage{pgfplots,pgfplotstable}

\makeatletter
\long\def\ifnodedefined#1#2#3{%
    \@ifundefined{pgf@sh@ns@#1}{#3}{#2}%
}

\pgfplotsset{
    discontinuous/.style={
    scatter,
    scatter/@pre marker code/.code={
        \ifnodedefined{marker}{
            \pgfpointdiff{\pgfpointanchor{marker}{center}}%
             {\pgfpoint{0}{0}}%
             \ifdim\pgf@y>0pt
                \tikzset{options/.style={mark=*, fill=white}}
                %\draw [densely dashed] (marker-|0,0) -- (0,0);
                \draw plot [mark=*] coordinates {(marker-|0,0)};
             \else
                \tikzset{options/.style={mark=none}}
             \fi
        }{
            \tikzset{options/.style={mark=none}}        
        }
        \coordinate (marker) at (0,0);
        \begin{scope}[options]
    },
    scatter/@post marker code/.code={\end{scope}}
    }
}

\newcommand*\Eval[3]{\left(#1\middle)\right\rvert_{#2}^{#3}}


\usepackage{arydshln}
\usepackage{mathtools}

%Accessibility issues
\usepackage[a-1b]{pdfx}

%\usepackage{tagpdf}
%\tagpdfsetup{activate,paratagging,interwordspace}
\usepackage{accsupp}
\usepackage{hyperref}
\usepackage{lipsum}
%\usepackage{axessibility}
\usepackage{accessibility}

%SetFonts
\DeclareEmphSequence{\bfseries\itshape}
\DeclareMathOperator {\ints}{\mathbb{Z}}
\DeclareMathOperator {\reals}{\mathbb{R}}
\DeclareMathOperator*{\argmin}{argmin}
\DeclareMathOperator*{\argmax}{argmax}


\renewcommand\labelenumi{(\theenumi)}
\renewcommand*{\thefootnote}{\fnsymbol{footnote}}

\usepackage{tcolorbox}

\newenvironment{aside}[1][Aside]{\begin{tcolorbox}[colback=black!5,colframe=black!75!black,title=#1]}{\end{tcolorbox}}

%SetFonts


\title{Topic 3: Graphs}
\author{02-680: Essentials of Mathematics and Statistics}
%\date{}							% Activate to display a given date or no date

\begin{document}
\maketitle

%%%
\section{The Basics}
Typically we denote a graph as the tuple $G=\langle V, E\rangle$, 
where $V$ is a set of nodes or \emph{vertices} and 
\[E\subseteq V\times V \textnormal{\hspace{3em} or \hspace{3em}} E\subseteq \left\{\left\{u,v\right\} : u,v \in V\right\}\] 
is a set of \emph{edges} or connections between two vertices.
Which definition we use is dependent on $G$ being \emph{directed} or \emph{undirected} (i.e. does the order of the nodes in the edge set matter).  

Some examples of things that can be represented as graphs: train/flight maps, cell interactions, social networks, etc. 
In computational biology, we also represent genomes as graphs, in this case as a directed graph with changes denoted by different \emph{paths} from start to finish. 
A \textit{path} is a sequence of nodes \[\langle v_1, v_2, v_3,...,v_k \rangle\] ($k\ge 1$) where every edge \[\langle v_{i}, v_{i+1}\rangle \in E\textnormal{\hspace{1em} (or  }\{ v_{i}, v_{i+1}\} \in E\textnormal{).}\]
We say the path has \emph{length} $k$, and that is a path \textit{from} $v_1$ \textit{to} $v_k$.

In the case of a genome graph, each edge would be \emph{labeled} with a string, but in other types of graphs the labels could be numerical 
(in which we would usually call them \textit{weights}).
A path label would then be the concatenation of all the edge labels, a path weight would be the sum of the weights. 
We can also label or weight edges in some scenarios. 
In all cases weights (labels) are typically represented by a function:
\[\ell: E \rightarrow \Sigma^* \]
(or maybe $w: E \rightarrow \reals$).

A lot of talked we want to talk about the \emph{neighborhood} of a node: 
\[N_G(v) = \left\{ u \mid \left\{u,v\right\} \in E\right\}\]
and we call the size of this set $\left|N(v)\right|$ the cardinality of the node or the \emph{degree}. 
In a directed graph we can restrict this to an in- and out-degree (and neighborhood)
\[N^{in}_G(v) = \left\{ u \mid \left\langle u,v\right\rangle \in E\right\}\]
\[N^{out}_G(v) = \left\{ u \mid \left\langle v,u\right\rangle \in E\right\}\]
and thus $N_G(v) = N^{in}_G(v) \cup N^{out}_G$ (and the degrees are the respective set cardinalities). 
We call all of the nodes in $N_G(v)$ \emph{adjacent} to $v$. 

A graph is \emph{regular} if \[\forall v \in V : |N_G(v)| = k\] for some fixed $k$. 

Note many times we will leave off the $G$ and simply write $N,N^{in},N^{out}$ when the graph in question is clear from context. 

%\paragraph{Isomorphism. } Not sure if this is necessary 

\paragraph{Complete Graphs.}
A complete graph is one where all edges are present. 
That is for $G=\langle V, E\rangle$ to be complete, 
\[E= V\times V \textnormal{\hspace{3em} or \hspace{3em}} E= \left\{\left\{u,v\right\} : u,v \in V\right\}\]
We also know in that case that it is $|V|-1$ regular.
We also call complete graphs \emph{Cliques}, especially in the context of subgraphs (below), 
and we do that so often we have a special notation for that: $\mathcal{K}_n$ (for a clique of size $n$). 

%\paragraph{Planar.}

\paragraph{Subgraph.}
Many times we only want to look at a particular part of a graph, 
think about say a regulatory network.
(A regulatory network is a directed graph with nodes that are genes, 
and edges going from one gene to another if the source somehow impacts the expression of the sink.) 
If we run an experiment and get a list of genes that are differentially expressed in some condition 
(i.e. the expression level is different from the null/healthy case) 
we may want to try and intuit something looking only at those changes. 

So we define a \emph{subgraph} $G' = \langle V', E'\rangle$ where 
$V' \subseteq V$ and $E' \subseteq \left\{\left\langle u,v\right\rangle \mid u,v\in V'\wedge \left\langle u,v\right\rangle \in E \right\}$.
That is we choose a subset of nodes, and the edges associated only with the chose nodes. 
In the example above, we almost always would want to look at the \emph{induced} subgraph, which is basically the same thing
but $E'$ is \textit{equal} to the set of edges above rather than a subset.

\subsection{Bipartite Graphs}
A \emph{Bipartite} graph is $G=\left\langle \left(A\cup B\right), E \right\rangle$ 
where \[E\subseteq \left\{\left\{u,v\right\} \mid u\in A, v\in B\right\}\] (in directed graphs edges can go from $A$ to $B$ or $B$ to $A$). 
That is, its a graph where the nodes can be separated into two groups, and no edge exists within the group.  
These graphs come up a lot when doing some sort of assignment, in which case the actual assignment is a subgraph with all of the nodes but some subset of edges.
An example could be assigning students to a peer advisor; 
maybe $A$ is the set of first year students, and $B$ is the second years. 
$E$ is $A\times B$, and the goal is to find a subgraph such that 
\[\forall v \in A: |N^{out}_{G'}(v)|=1\]
(the out-degree of each node in the subgraph is 1).

%%Connected Components? 

\paragraph{Connected.} 


%%%%%%%%%
\section{Trees}
A tree is a special type of graph that is fully connected and has exactly $|V|-1$ edges. 

One consequence of this is that there are no \emph{cycles};
a cycle is a special type of path $\langle v_1, v_2,.... v_{k-1}, v_1\rangle$ ($k\ge 2$) that ends at the same node it started at. 

%Using one set and one function, define a tree (remember that $f(x)$ can only have one value...

\paragraph{Phylogeny.}


\section*{Useful References}
Liben-Nowell, ``Connecting Discrete Mathematics and Computer Science, 2e''. \S 11.2-11.4

\end{document}