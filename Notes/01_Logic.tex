\documentclass[11pt, oneside]{article}   	% use "amsart" instead of "article" for AMSLaTeX format
\documentclass[11pt, oneside]{article}   	% use "amsart" instead of "article" for AMSLaTeX format

\usepackage{geometry}                		% See geometry.pdf to learn the layout options. There are lots.
\geometry{letterpaper}                   		% ... or a4paper or a5paper or ... 
%\geometry{landscape}                		% Activate for rotated page geometry
\usepackage[parfill]{parskip}    		% Activate to begin paragraphs with an empty line rather than an indent
\usepackage{graphicx}				% Use pdf, png, jpg, or eps§ with pdflatex; use eps in DVI mode
								% TeX will automatically convert eps --> pdf in pdflatex		
\usepackage{amssymb}
\usepackage{amsmath}
\usepackage{tablefootnote}
\usepackage{multirow}
\usepackage{tabularx}
\renewcommand\tabularxcolumn[1]{m{#1}}

\usepackage{dsfont}
\usepackage[hang,flushmargin]{footmisc}

\usepackage{pgfplots,pgfplotstable}

\makeatletter
\long\def\ifnodedefined#1#2#3{%
    \@ifundefined{pgf@sh@ns@#1}{#3}{#2}%
}

\pgfplotsset{
    discontinuous/.style={
    scatter,
    scatter/@pre marker code/.code={
        \ifnodedefined{marker}{
            \pgfpointdiff{\pgfpointanchor{marker}{center}}%
             {\pgfpoint{0}{0}}%
             \ifdim\pgf@y>0pt
                \tikzset{options/.style={mark=*, fill=white}}
                %\draw [densely dashed] (marker-|0,0) -- (0,0);
                \draw plot [mark=*] coordinates {(marker-|0,0)};
             \else
                \tikzset{options/.style={mark=none}}
             \fi
        }{
            \tikzset{options/.style={mark=none}}        
        }
        \coordinate (marker) at (0,0);
        \begin{scope}[options]
    },
    scatter/@post marker code/.code={\end{scope}}
    }
}

\newcommand*\Eval[3]{\left(#1\middle)\right\rvert_{#2}^{#3}}


\usepackage{arydshln}
\usepackage{mathtools}

%Accessibility issues
\usepackage[a-1b]{pdfx}

%\usepackage{tagpdf}
%\tagpdfsetup{activate,paratagging,interwordspace}
\usepackage{accsupp}
\usepackage{hyperref}
\usepackage{lipsum}
%\usepackage{axessibility}
\usepackage{accessibility}

%SetFonts
\DeclareEmphSequence{\bfseries\itshape}
\DeclareMathOperator {\ints}{\mathbb{Z}}
\DeclareMathOperator {\reals}{\mathbb{R}}
\DeclareMathOperator*{\argmin}{argmin}
\DeclareMathOperator*{\argmax}{argmax}


\renewcommand\labelenumi{(\theenumi)}
\renewcommand*{\thefootnote}{\fnsymbol{footnote}}

\usepackage{tcolorbox}

\newenvironment{aside}[1][Aside]{\begin{tcolorbox}[colback=black!5,colframe=black!75!black,title=#1]}{\end{tcolorbox}}

%SetFonts


\title{Topic 3:  Logic}
\author{02-680: Essentials of Mathematics and Statistics}
%\date{}							% Activate to display a given date or no date

\begin{document}
\maketitle

%%%%%
\section{Propositions}
In logic a \emph{proposition} is simply a statement that can be evaluated for truth. 
Something like ``$2+2=4$'' or ``DJ is the CMU President''. 
We usually use lower-case letters to represent \emph{atomic propositions}, sort of like variables. 
Something like 
\[ p \leftarrow \textnormal{``}2+2=4\textnormal{''}\]
\[ q \leftarrow \textnormal{``DJ is the CMU President''}.\]

We know that $p$ is true, and $q$ is false. 

A \emph{compound proposition} can take into account multiple atopic propositions to create a single statement: 
\[ p \wedge q \]
(which doesn't need to be true). 
We read the previous as ``$p$ \textit{and} $q$'', 
so for the whole statement to be true both atomic elements need to be true. 
 
But we can also \textit{negate} a proposition: 
\[ p \wedge \neg q \]
now the statement is true! 
We read this as ``$p$ and not $q$'', 
or  ``$2+2=4$'' and  ``DJ is \emph{not} the CMU President''.

All of the connectives and operations are listed below: 
\begin{center}
\begin{tabular}{lll}
name 		& symbol/use 		& description\\
\hline 
\hline
negation			& $\neg p$		& ``not $p$''\\
\hline
conjunction		& $p \wedge q$		& ``$p$ and $q$''\\
disjunction		& $p \vee q$		& ``$p$ or $q$''\\
exclusive or		& $p \oplus q$		& ``$p$ or $q$ but not both''\\
\hline
implication		& $p \implies q$	& ``if $p$, then $q$''\\
double implication 	& $p \iff q$		& ``$p$ if and only if $q$'' \\
				&				& \hspace{3em}(sometimes shortened to ``iff'')\\
\hline
\end{tabular}
\end{center}

While the order of operations on propositions is top to bottom in the table above, 
its best to use parenthesis to make sure your statements are clear. 
Technically the two statements below are the same, one is much more clear:
\[
p \vee q \implies \neg r \wedge \ell \;\;\;\; \textnormal{and}\;\;\;\; \left(p\vee q\right) \implies \left(\neg r \wedge \ell\right) 
\]

%%%%%
\section{Predicate Logic}
We often need to talk about groups of items;
for example we have already talked about sets, so can we make logical statements about sets (or elements there of). 

For example, if we want to say something like:
``for any integer $x$, the value of $x * 0$ is $0$.''
We would write that using the \emph{universal qualifier}:
\[\forall x\in\ints : x*0 = 0\]
(we say $\forall$ as ``for all'', in fact the latex command for the symbol is \texttt{\textbackslash{}forall}).

But sometimes we want to not talk about all items, but one (or more) in particular, in which case we can use the \emph{existential qualifier}:
\[\exists y\in\ints: y^2 = |y+y|\]
(we say $\exists$ as ``there exists'', and the latex is \texttt{\textbackslash{}exists}). 
In this case there exists \textit{more than one} $y$ that satisfies the proposition, so the statement is true.

We already saw several examples of these symbols in earlier discussions. 

 \subsection{Nested Qualifiers}

\end{document}