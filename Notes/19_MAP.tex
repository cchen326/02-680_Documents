\documentclass[11pt, oneside]{article}   	% use "amsart" instead of "article" for AMSLaTeX format

\usepackage{geometry}                		% See geometry.pdf to learn the layout options. There are lots.
\geometry{letterpaper}                   		% ... or a4paper or a5paper or ... 
%\geometry{landscape}                		% Activate for rotated page geometry
\usepackage[parfill]{parskip}    		% Activate to begin paragraphs with an empty line rather than an indent
\usepackage{graphicx}				% Use pdf, png, jpg, or eps§ with pdflatex; use eps in DVI mode
								% TeX will automatically convert eps --> pdf in pdflatex		
\usepackage{amssymb}
\usepackage{amsmath}
\usepackage{tablefootnote}
\usepackage{multirow}
\usepackage{tabularx}
\renewcommand\tabularxcolumn[1]{m{#1}}

\usepackage{dsfont}
\usepackage[hang,flushmargin]{footmisc}

\usepackage{pgfplots,pgfplotstable}

\makeatletter
\long\def\ifnodedefined#1#2#3{%
    \@ifundefined{pgf@sh@ns@#1}{#3}{#2}%
}

\pgfplotsset{
    discontinuous/.style={
    scatter,
    scatter/@pre marker code/.code={
        \ifnodedefined{marker}{
            \pgfpointdiff{\pgfpointanchor{marker}{center}}%
             {\pgfpoint{0}{0}}%
             \ifdim\pgf@y>0pt
                \tikzset{options/.style={mark=*, fill=white}}
                %\draw [densely dashed] (marker-|0,0) -- (0,0);
                \draw plot [mark=*] coordinates {(marker-|0,0)};
             \else
                \tikzset{options/.style={mark=none}}
             \fi
        }{
            \tikzset{options/.style={mark=none}}        
        }
        \coordinate (marker) at (0,0);
        \begin{scope}[options]
    },
    scatter/@post marker code/.code={\end{scope}}
    }
}

\newcommand*\Eval[3]{\left(#1\middle)\right\rvert_{#2}^{#3}}


\usepackage{arydshln}
\usepackage{mathtools}

%Accessibility issues
\usepackage[a-1b]{pdfx}

%\usepackage{tagpdf}
%\tagpdfsetup{activate,paratagging,interwordspace}
\usepackage{accsupp}
\usepackage{hyperref}
\usepackage{lipsum}
%\usepackage{axessibility}
\usepackage{accessibility}

%SetFonts
\DeclareEmphSequence{\bfseries\itshape}
\DeclareMathOperator {\ints}{\mathbb{Z}}
\DeclareMathOperator {\reals}{\mathbb{R}}
\DeclareMathOperator*{\argmin}{argmin}
\DeclareMathOperator*{\argmax}{argmax}


\renewcommand\labelenumi{(\theenumi)}
\renewcommand*{\thefootnote}{\fnsymbol{footnote}}

\usepackage{tcolorbox}

\newenvironment{aside}[1][Aside]{\begin{tcolorbox}[colback=black!5,colframe=black!75!black,title=#1]}{\end{tcolorbox}}


\title{Topic 19: Maximum \emph{a posteriori} Estimation}
\author{02-680: Essentials of Mathematics and Statistics}
%\date{}							% Activate to display a given date or no date

\begin{document}
\maketitle

%%%%%%%%%%%%%%%
\section{The Frequentist versus Bayesian Schools}

Both schools of statistics start with probability. 
For Bayesian inference, we take $H$ to be a hypothesis (parameters) and $D$ some data. 
Different people will have different a priori beliefs --- 
but we would still like to make useful inferences from data. 

When $p(H)$ is known, there is no disagreement, we will all just follow Bayes' Rule as written. 

\begin{center}
\begin{dot2tex}[dot,mathmode]
digraph O {
direction="TB";
node [shape="box"];
B [label="p(H\mid D) = \frac{p(D\mid H)p(H)}{p(D)}"];
P [label="p(H\mid D) = \frac{p(D\mid H){\color{blue} p_{prior}(H)}}{p(D)}"];
F [label="\text{Likelihood }L(H \mid D) = p(D\mid H)"];
B->P[color="blue",label="\color{blue}\text{Bayesian}"];
B->F[color="orange",label="\color{orange}\text{Frequentist}"];
}
\end{dot2tex}
\end{center}

When the prior is \textit{not} known, 
\emph{Bayesian} logic requires us to develop a prior based on some information we have (intuition); 
while \emph{Frequentist} logic uses only information in the provided data. 

Bayesians and frequentists take fundamentally different approaches to this challenge.
The reasons for this split are both practical (ease of implementation and computation) and 
philosophical (subjectivity versus objectivity and the nature of probability). 

The main philosophical difference concerns the \emph{meaning of probability}.

\begin{tabularx}{\textwidth}{XX}
\hline
Bayesians & Frequentists\\
\hline\hline
the idea that probability is an abstract concept that measures a state of knowledge or a degree of belief in a given proposition &
the idea that probabilities represent long-term frequencies of repeatable random experiments\\
\hline
Subjective interpretation
& Objective interpretation\\
\hline
you ``believe'' that you will get tails 50\% of the time & 
the relative frequency of tails goes to 1/2 as the number of flips goes to infinity\\
they consider a range of values each with its own probability of being true
\end{tabularx}

%%%%%%%%%%%%%
\section{Bayesians' Approach to Parameter Estimation}
Lets look at the coin flip example we had in the last topic: 
 $\mathcal{D} = X_1, X_2, \cdots, X_n$ 
where $X_i \sim Bernouli(\alpha)$.
We can further summarize $\mathcal{D}$ into $c_H$ and $c_T$ 
representing the counts of heads and tails respectively. 

We saw last time that \[\hat\alpha_{MLE} = \frac{c_H}{c_H+c_T}.\]

But this is assuming we know nothing about $\alpha$ ahead of time. 
What if we \textit{believe} that its 50/50, so we can add what are called \emph{pseudocounts}
to the input $c_{H_0}$ and $c_{T_0}$.
And thus compute 
\[\hat\alpha_{MLE-PC} = \frac{c_H+c_{H_0}}{c_H+c_T+c_{H_0}+c_{T_0}}.\]

\paragraph{Example. }
Lets assume we have some an experiment where we throw a coin 100 times, we want to know $\alpha$, $c_H=0$ and $c_T=100$.

Vanilla MLE would say that 
\[\hat\alpha_{MLE} = \frac{c_H}{c_H+c_T} = \frac{0}{0 + 100} = 0\]
  
But we have a small belief this is a fair coin, so lets assume we add the pseudocounts $c_{H_0}=c_{T_0}=1$, 
in that case 
\[\hat\alpha_{MLE-PC} = \frac{c_H+c_{H_0}}{c_H+c_T+c_{H_0}+c_{T_0}}=\frac{0+1}{0+100+1+1} = \frac{1}{102}.\]

If we're more confident in our prior and set  $c_{H_0}=c_{T_0}=100$ then
\[\hat\alpha_{MLE-PC} = \frac{c_H+c_{H_0}}{c_H+c_T+c_{H_0}+c_{T_0}}=\frac{0+100}{0+100+100+100} = \frac{100}{300} = \frac{1}{3}.\]

Pseudocounts are a way of exerting your \textit{belief}
\begin{itemize}
\item[] \textbf{Larger pseudocounts} – represent a strong prior belief\\\hspace*{3em}
Will have a greater effect on the posterior estimate
\item[] \textbf{Small pseudocounts} – represent a weak prior belief\\\hspace*{3em}
Will have a smaller effect on the posterior estimate
\end{itemize}

As the sample size goes to infinity, data will dominate the estimate.

\end{document}