%%%%%%%%%%%%%%%%%%%%%%%%%%%%%%%%%%%%
%%%%%%%%%%%%%%%%%%%%%%%%%%%%%%%%%%%%
\section{Resources}
%%%%%%%%%%%%%%%%%%%%%%%%%%%%%%%%%%%%
%%%%%%%%%%%%%%%%%%%%%%%%%%%%%%%%%%%%

\subsection{Accommodations for Students with Disabilities}

If you have a disability and have an accommodations letter from the Disability Resources office, 
we encourage you to discuss your accommodations and needs with me as early in the semester as possible. 
We will work with you to ensure that accommodations are provided as appropriate. 
If you suspect that you may have a disability and would benefit from accommodations 
but are not yet registered with the Office of Disability Resources, 
we encourage you to contact them at access@andrew.cmu.edu.

\subsection{Statement of Support for Students’ Health \& Well-being}

Take care of yourself.  
Do your best to maintain a healthy lifestyle this semester by eating well, exercising, 
avoiding drugs and alcohol, getting enough sleep and taking some time to relax. 
This will help you achieve your goals and cope with stress.
All of us benefit from support during times of struggle. 
There are many helpful resources available on campus and an important part of the college experience 
is learning how to ask for help. Asking for support sooner rather than later is almost always helpful.
If you or anyone you know experiences any academic stress, difficult life events, or feelings like anxiety or depression,
we strongly encourage you to seek support. 
Counseling and Psychological Services (CaPS) is here to help: 
call 412-268-2922 and visit their website at \url{http://www.cmu.edu/counseling/}. 
Consider reaching out to a friend, faculty or family member you trust 
for help getting connected to the support that can help.

 \begin{tcolorbox}[colback=blue!5,colframe=blue!75!black]
If you or someone you know is feeling suicidal or in danger of self-harm, call someone immediately, day or night:\\

CaPS: 412-268-2922\\
Re:solve Crisis Network: 888-796-8226\\
Suicide and Crisis Lifeline: 988\\ 

If the situation is life threatening, call the police\\
On campus: CMU Police: 412-268-2323\\
Off campus: 911
\end{tcolorbox}

\subsection{Diversity Statement}

We must treat every individual with respect. 
We are diverse in many ways, 
and this diversity is fundamental to building and maintaining an equitable and inclusive campus community. 
Diversity can refer to multiple ways that we identify ourselves, 
including but not limited to 
race, color, national origin, language, sex, disability, age, sexual orientation, 
gender identity, religion, creed, ancestry, belief, veteran status, or genetic information. 
Each of these diverse identities, 
along with many others not mentioned here, shape the perspectives our students, faculty, and staff bring to our campus.
We, at CMU, will work to promote diversity, equity and inclusion 
not only because diversity fuels excellence and innovation, 
but because we want to pursue justice. 
We acknowledge our imperfections while we also fully commit to the work, 
inside and outside of our classrooms, 
of building and sustaining a campus community that increasingly embraces these core values.

Each of us is responsible for creating a safer, more inclusive environment.

Unfortunately, incidents of bias or discrimination do occur, whether intentional or unintentional. 
They contribute to creating an unwelcoming environment for individuals and groups at the university. 
Therefore, the university encourages anyone who experiences or observes unfair or hostile treatment 
on the basis of identity to speak out for justice and support, 
within the moment of the incident or after the incident has passed. 
Anyone can share these experiences using the following resources:

\textbf{Center for Student Diversity and Inclusion:} csdi@andrew.cmu.edu, (412) 268-2150

\textbf{Report-It online anonymous reporting platform:} reportit.net username: tartans password: plaid

All reports will be documented and deliberated to determine if there should be any following actions. 
Regardless of incident type, the university will use all shared experiences to transform our campus climate to be more equitable and just.



